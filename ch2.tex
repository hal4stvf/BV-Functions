\newcommand{\weakstar}{\overset{*}{\rightharpoonup}}
\section{Covering theorems and differentiation of measures}
\subsection{Mollification of Radon measure}

Let $\mu \in \mathcal{M}_{loc}(\Omega;\R^m)$ and $g \in C_c(\R^n)$.
We denote by $(\mu * g)$ the convolution :
\[
(\mu * g)(x) := \int_\Omega g(x-y) d\mu(y)
\quad \forall x:
\quad \text{the map $\quad y \mapsto g(x-y)$ is $C_c$}
\]
In particular, let $\rho \in C^\infty_c (B(0,1))$, $\rho(-x)= \rho(x)$, $\rho
\geq 0$, $\int_{\R^n} \rho dx  = 1$ ($\rho$ is a mollifier), set
$\rho_\varepsilon(x) : = \frac{1}{\varepsilon^n} \rho(\frac{x}{\varepsilon})$,
then  
\[
(\mu * \rho_\varepsilon)(x) = \int_\Omega \rho_\varepsilon(x-y) d\mu(y)
\]
is well defined for all $x \in \Omega^\varepsilon := \{x \in \Omega:
\dist(x,\partial \Omega) > \varepsilon\}$. 

\begin{theorem}[Properties of mollifications]
Let $\mu \in \mathcal{M}_{loc}(\Omega;\R^m)$, $\rho$ a mollifier. Then $\mu *
\rho_\varepsilon \in C^\infty(\Omega^\varepsilon;\R^m)$, $D^\alpha(\mu *
\rho_\varepsilon) = \mu * D^\alpha \rho_\varepsilon$ (for all $\alpha \in
\N_0^n$). In addition
\[
\mu * \rho_\varepsilon \weakstar \mu 
\]
for all $E$ Lebesgue measure, $\int_E |\mu * \rho_\varepsilon|dx \leq
	|\mu|(E_\varepsilon)$ for all $\varepsilon >0$.
\end{theorem}
\begin{proof}
Regularity and differentiability is an exercise.
Let $\varphi \in C_c(\Omega;\R^m)$
\[
\begin{aligned}
\int_\Omega \varphi \cdot (\mu * \rho_\varepsilon) dx 
= \int_\Omega \int_\Omega \varphi(x) \rho_\varepsilon(x-y) \cdot d\mu(y) dx 
= \int_\Omega (\varphi * \rho_\varepsilon)(y) \cdot d\mu(y) \to \int \varphi
\cdot d\mu(y),
\end{aligned}
\]
where we used $\rho_\varepsilon(x-y) = \rho_\varepsilon(y-x)$ and
$(\varphi*\rho_\varepsilon)(y) \to \varphi(y)$ uniformly.
By Fubini's Theorem,
\TODO
\end{proof}

\subsection{Differentiation of Radon and Hausdorff measures}
We say that a family $\mathcal{F}$ is disjoint if $F \cap F' = \emptyset$ for
all $F,F' \in \mathcal{F}$, $F \neq F'$.

\begin{theorem}[Besicovitch covering theorem]
There exists a $\xi_n \in \N$ such that for all families of closed balls
$\mathcal{F}$ such that the set 
$A: = \{ x \in \R^n \mid \exists \varrho > 0 : \overline{B(x,\varrho)} \in
\mathcal{F}\}$ is bounded, there exists at mos $\xi_n$ disjoint subfamilies
$\mathcal{F}_i \subset \mathcal{F}$ such that 
\[
A \subset \bigcup_{i=1}^{\xi_n} \bigcup_{\overline{B} \in \mathcal{F}_i}
\overline{B}.
\]
\end{theorem}

\begin{remark}
It works also with open balls. 
\end{remark}

\begin{theorem}[consequence of Besicovitch]
Let $A$ be a bounded set and $\varrho : A \to (0,\infty)$. Then there exists $S
\subset A$ at most countable such that 
\(
A \subset \bigcup_{x\in S} B(x,\varrho(x)) 
\)
and for all $y \in \R^n$
\[
\sum_{x \in S} \chi_{B(x,\varrho(x))}(y) \leq \xi_n.
\]
\end{theorem}
\begin{proof}
Let $\mathcal{F} := \{(B(x,\varrho(x)) \mid x \in A\}$ \TODO
\end{proof}

\begin{definition}
Let $\mathcal{F}$ be a family of closed balls and $A \subset \R^n$.
$\mathcal{F}$ is called a fine covering of $A$ if 
\[
\inf \left\{ \varrho > 0 \mid \overline{B(x,\varrho)} \in \mathcal{F}\right \} =
0 \quad \text{for all $x\in A$}.
\]
\end{definition}

\begin{theorem}[Vitali]
Let $A$ be a bounded Borel set and $\mathcal{F}$ be a fine covering of $A$. In
any case $\mu \in \mathcal{M}^+_{loc}(\R^n)$ there exists $\mathcal{F}' \subset
\mathcal{F}$ disjoint such that 
\[
\mu \left( A \setminus \bigcup_{\overline{B}
\in \mathcal{F}'} \overline{B} \right) = 0
\]
\end{theorem}
\begin{proof}
\TODO
\end{proof}

\begin{theorem}
let $\mu \in \mathcal{M}^+_{loc}(\Omega)$, $\lambda \in
\mathcal{M}_{loc}(\Omega;\R^m)$, $|\lambda| << \mu$. Then, for $\mu$-a.e. $x \in
\supp u$, the limit 
\[
\lim_{\varrho \to 0} \frac{\lambda(B(x,\varrho))}{\mu(B(x,\varrho))}
\]
exists in $\R^n$.
\end{theorem}

By Radon-Nikodyn, $\lambda = f\mu$ for some $f \in
\Leb{1}_{\text{loc}}(\Omega;\mu)$. Thus 
\[
f(x) = \lim_{\varrho \to 0} \frac{\lambda(B(x,\varrho))}{\mu(B(x,\varrho))}
\]
for $\mu$-a.e. $x \in \supp \mu$.



\section{Fine properties of Lipschitz functions}

We devote this section to the discussion of some properties of Lipschitz functions, which proved to be very useful in the framework of Geometric Measure Theory. 
The choice of working with Lipschitz functions is due to the fact that such functions have a less rigid structure than $C^{1}$-differentiable functions (for instance, extension theorems are much easier to prove, see McShane's lemma), while they enjoy differentiability properties almost everywhere (see Rademacher's theorem).

\begin{lemma}[McShane's lemma] \label{McShane_lemma}
Let $E \subset \R^{n}$ and $f: E \to \R$ be a Lipschitz function. Then the function $f^{+} : \R^{n} \to \R$ defined as
\begin{equation*}
f^{+}(x) := \inf \{ f(y) + \Lip(f, E) |x - y| : y \in E\}
\end{equation*}
is Lipschitz and it satisfies $f^{+}(x) = f(x)$ for any $x \in E$ and $\Lip(f, E) = \Lip(f^{+}, \R^{n})$.
\end{lemma}
\begin{proof}
For any $x, z \in \R^{n}$, by the triangle inequality, we have
\begin{equation*}
f^{+}(x) \le \inf \{ f(y) + \Lip(f, E) (|x - z| + |z - y|) : y \in E \} = f^{+}(z) + \Lip(f, E) |x - z|.
\end{equation*}
Then, interchanging the role of $x$ and $z$, we immediately get
\begin{equation*}
|f^{+}(x) - f^{+}(z)| \le \Lip(f, E) |x - z|.
\end{equation*}
Finally, let $x \in E$. It is easy to see that $f^{+}(x) \le f(x)$. In order to obtain the reverse inequality, notice that 
\begin{equation*}
f(x) \le f(y) + \Lip(f, E)|x - y|
\end{equation*}
for any $y \in E$, since $f$ is Lipschitz on $E$.
\end{proof}

\begin{remark} The extension given in McShane's lemma is the largest extension of $f$, while, arguing analogously, one can show that the smaller extension is given by
\begin{equation*}
f^{-}(x) := \sup \{ f(y) - \Lip(f, E) |x - y| : y \in E \}.
\end{equation*}
\end{remark}

It is not difficult to see that McShane's lemma can be extended to vector valued Lipschitz functions by hands; however, in such a way we loose the equality between the Lipschitz constants.

\begin{corollary}
Let $E \subset \R^{n}$ and $f: E \to \R^{m}$ be a Lipschitz function. Then there exists a Lipschitz function $\tildef{f} : \R^{n} \to \R^{m}$ such that $\tildef{f} = f$ on $E$ and $\Lip(\tildef{f}, \R^{n}) \le \sqrt{m} \Lip(f, E)$.
\end{corollary}
\begin{proof}
Apply McShane's lemma (Lemma \ref{McShane_lemma}) to each component of $f$, thus defining
\begin{equation*}
\tildef{f} := (f_{1}^{+}, \dots, f_{m}^{+}).
\end{equation*}
Then it is easy to see that $\tildef{f}= f$ on $E$. As for the Lipschitz constant, notice that
\begin{equation*}
|\tildef{f}(x) - \tildef{f}(y)|^{2} = \sum_{i = 1}^{m}|f_{i}^{+}(x) - f_{i}^{+}(y)|^{2} \le m (\Lip(f, E))^{2} |x - y|^2.
\end{equation*}
This ends the proof.
\end{proof}

A more refined result was found by Kirszbraun.

\begin{theorem}[Kirszbraun theorem]
Let $E \subset \R^{n}$ and $f: E \to \R^{m}$ be a Lipschitz function. Then there exists a Lipschitz function $g: \R^{n} \to \R^{m}$ such that $g = f$ on $E$ and $\Lip(g, \R^{n}) = \Lip(f, E)$.
\end{theorem}

A practical consequence of these extension results for Lipschitz functions is that we may always assume, without loss of generality, that our Lipschitz maps are defined on the whole space $\R^{n}$.

We shall now see that, quite surprisingly, the Lipschitz continuity property is enough to ensure differentiability outside of a Lebesgue negligible set. We start by recalling the notion of differentiability.

\begin{definition}
A function $f : \R^{n} \to \R^{m}$ is {\em differentiable} at $x \in \R^{n}$ if there exists a linear mapping $L: \R^{n} \to \R^{m}$ such that
\begin{equation*}
\lim_{y \to x} \frac{|f(y) - f(x) - L(y - x)|}{|y - x|} = 0.
\end{equation*}
This linear mapping is denoted by $\nabla f(x)$ or $d f(x)$.
\end{definition}

\begin{theorem}[Rademacher's theorem]
Let $f : \R^{n} \to \R^{m}$ be a locally Lipschitz function. Then $f$ is differentiable at $\Leb{n}$-a.e. $x \in \R^{n}$. In particular, $\nabla f(x)$ is well defined for $\Leb{n}$-a.e. $x \in \R^{n}$ and belongs to $L^{\infty}_{\rm loc}(\R^{n}; \R^{m} \times \R^{n})$, with $$\|\nabla f\|_{L^{\infty}(K; \R^{m} \times \R^{n})} \le \Lip(f, K)$$ for any compact set $K$.
\end{theorem}

An interesting consequence of this result is that the differential of a Lipschitz function vanishes on the level sets of the function.

\begin{theorem}
Let $f: \R^{n} \to \R^{m}$ be locally Lipschitz and $t \in \R$. Then $\nabla f(x) = 0$ for $\Leb{n}$-a.e. $x \in \{ f = t\} := \{ y \in \R^{n}: f(y) = t\}$.
\end{theorem}

\section{The area and coarea formulas}


\subsection{Linear maps and Jacobians}

We recall here some standard definitions and facts from linear algebra.

\begin{definition} \hfill
\begin{enumerate}[i)]
\item A linear map $O : \R^{n} \to \R^{m}$ is {\em orthogonal} if $$(Ox) \cdot (Oy) = x \cdot y$$ for all $x, y \in \R^{n}$.
\item A linear map $S: \R^{n} \to \R^{m}$ is {\em symmetric} if $$x \cdot (Sy) = (Sx) \cdot y$$ for all $x, y \in \R^{n}$.
\item Let $A: \R^{n} \to \R^{m}$. The {\em adjoint} of $A$ is the linear map $A^{*}: \R^{m} \to \R^{n}$ defined by $$x \cdot (A^{*} y) = (Ax) \cdot y$$ for all $x \in \R^{n}, y \in \R^{m}$. 
\end{enumerate}
\end{definition}

\begin{proposition} \label{prop:properties_linear} \hfill
\begin{enumerate}[i)]
\item Let $A: \R^{n} \to \R^{m}$ and $B: \R^{k} \to \R^{n}$ be linear maps. Then we have $A^{**} = A$ and $(A \circ B)^{*} = B^{*} \circ A^{*}$.
\item Let $S : \R^{n} \to \R^{n}$ be a symmetric linear map. Then $S^{*} = S$.
\item If $O : \R^{n} \to \R^{m}$ is an orthogonal linear map, then $n \le m$ and 
\begin{align*}
O^{*} \circ O = I & \text{ on } \R^{n}, \\
O \circ O^{*} = I & \text{ on } \R^{m}.
\end{align*}
\end{enumerate}
\end{proposition}

\begin{theorem}[Polar decomposition] \label{thm:polar_decomposition_map}
Let $L: \R^{n} \to \R^{m}$ be a linear mapping.
\begin{enumerate}[i)]
\item If $n \le m$, there exists a symmetric map $S: \R^{n} \to \R^{m}$ and an orthogonal map $O: \R^{n} \to \R^{m}$ such that $$L = O \circ S.$$
\item If $n \ge m$, there exists a symmetric map $S: \R^{m} \to \R^{m}$ and an orthogonal map $O: \R^{m} \to \R^{n}$ such that $$L = S \circ O^{*}.$$
\end{enumerate}
\end{theorem}

\begin{definition}[Jacobian]
Assume $L : \R^{n} \to \R^{m}$ is linear.
\begin{enumerate}[i)]
\item If $n \le m$, we write $L = O \circ S$ as above, and we define the {\em Jacobian} of $L$ as $$\J L := |\det{S}|.$$ 
\item If $n \le m$, we write $L = S \circ O^{*}$ as above, and we define the {\em Jacobian} of $L$ as $$\J L := |\det{S}|.$$ 
\end{enumerate}
\end{definition}

In the literature, these two different definitions of Jacobian are also called {\em $n$-dimensional Jacobian} (or {\em area factor}), and {m}-dimensional {\em coarea factor}, respectively, and are denoted by $\J_{n}$ and ${\bf C}_{m}$.

\begin{theorem}[Representation of Jacobian] \hfill
\begin{enumerate}[i)]
\item If $n \le m$, $$\J L = \sqrt{\det{(L^{*} \circ L)}}.$$
\item If $n \ge m$, $$\J L = \sqrt{\det{(L \circ L^{*})}}.$$
\end{enumerate}
\end{theorem}
\begin{proof}
Let $n \le m$ and $L = O \circ S$, by Theorem \ref{thm:polar_decomposition_map}. Then we have $L^{*} = S \circ O^{*}$, so that $$L^{*} \circ L = S \circ O^{*} \circ O \circ S = S^{2},$$ since $O$ is orthogonal and so $O^{*} \circ O$ is the identity mapping on $\R^{n}$ (by Proposition \ref{prop:properties_linear}). Hence $$\det{(L^{*} \circ L)} = \det{S^{2}} = (\J L)^{2}.$$
The proof of (ii) is similar.
\end{proof}

\begin{remark}
The definition of the Jacobian of $L$ is independent of the choices of $O$ and $S$, and we have $\J L = \J L^{*}$.
\end{remark}

\begin{proposition}[Cauchy-Binet formula] \label{Cauchy_Binet}
If $n \le m$ and $L: \R^{n} \to \R^{m}$ is a linear map, then
\begin{equation*}
\J L = \sqrt{ \sum_{B} \det^{2}{(B)}}
\end{equation*}
where the sum is taken over all $n\times n$ minor of any matrix representation of $L$.
\end{proposition}

Let now $f : \R^{n} \to \R^{m}$, $f = (f^{1}, \dots, f^{m}),$ be a Lipschitz map. By Rademacher's theorem, $f$ is differentiable $\Leb{n}$-a.e. and therefore the gradient matrix 
\[\nabla f(x) = \begin{pmatrix} \frac{\partial f^{1}}{\partial x_{1}} & \cdots & \frac{\partial f^{1}}{\partial x_{n}} & \\ \vdots & \ddots & \vdots \\  \frac{\partial f^{m}}{\partial x_{1}} & \cdots & \frac{\partial f^{m}}{\partial x_{n}} & \end{pmatrix}\]
is well defined and can be considered a linear map from $\R^{n}$ into $\R^{m}$ for $\Leb{n}$-a.e. $x \in \R^{n}$.

\begin{definition}
If $f : \R^{n} \to \R^{m}$ is Lipschitz continuous and $x$ is a differentiability point, we define the {\em Jacobian} of $f$ as $$\J f(x) := \J \nabla f(x).$$
\end{definition}

\begin{remark}
Notice that $\J f \le c_n \Lip(f)^{n}$.
\end{remark}

\subsection{The area formula}

Through this subsection we assume $n \le m$ and $f : \R^{n} \to \R^{m}$ to be Lipschitz continuous.

\begin{lemma} Let $A \subset \R^{n}$ be $\Leb{n}$-measurable. Then
\begin{enumerate}[i)]
\item $f(A)$ is $\Haus{n}$-measurable,
\item the mapping $y \to \Haus{0}(A \cap f^{-1}(y))$ is $\Haus{n}$-measurable on $\R^{m}$ and
\begin{equation*}
\int_{\R^{m}} \Haus{0}(A \cap f^{-1}(y)) \, d \Haus{n}(y) \le (\Lip(f))^{n} \Leb{n}(A)
\end{equation*}
\end{enumerate}
\end{lemma}

\begin{definition}
The mapping $y \to \Haus{0}(A \cap f^{-1}(y))$ is the {\em multiplicity function} of $f$ in $A$.
\end{definition}

\begin{remark}
It is easy to notice that $\Haus{0}(A \cap f^{-1}(y))$ is equal to the cardinality of the set of $$\{ x \in A : f(x) = y \},$$
so that $f^{-1}(y)$ is finite for $\Haus{n}$-a.e. $y \in \R^{m}$.
In particular, if $f$ is injective, then
\begin{equation*}
\Haus{0}(A \cap f^{-1}(y)) = \begin{cases} 1 & y \in f(A), \\
0 & y \notin f(A).
\end{cases}
\end{equation*}
\end{remark}

\begin{theorem}[Area formula] \label{area_formula}
Let $f : \R^{n} \to \R^{m}$ be Lipschitz continuous and $n \le m$. Then, for all $\Leb{n}$-measurable sets $A \subset \R^{n}$, we have
\begin{equation} \label{eq:area_formula}
\int_{A} \J f(x) \, dx = \int_{\R^{m}} \Haus{0}(A \cap f^{-1}(y)) \, d\Haus{n}(y).
\end{equation}
\end{theorem}

This means that the $\Haus{n}$-measure of $f(A)$, counting multiplicity, is equal to the integral of the Jacobian of $f$ over $A$. As an immediate consequence, we deduce a generalization of the classical change of variables formula.

\begin{theorem}[General change of variables] \label{change_variables_gen}
Let $f : \R^{n} \to \R^{m}$ be Lipschitz continuous and $n \le m$. Then, for all $\Leb{n}$-summable functions $g : \R^{n} \to \R$, we have
\begin{equation} \label{eq:change_variables_gen}
\int_{\R^{n}} g(x) \, \J f(x) \, dx = \int_{\R^{m}} \left (\sum_{x \in f^{-1}(y)} g(x) \right ) d \Haus{n}(y).
\end{equation}
\end{theorem}

\begin{corollary}[Injective maps] \label{change_variables}
Let $f : \R^{n} \to \R^{m}$ be Lipschitz continuous and $n \le m$. Let $g : \R^{n} \to \R$ be a $\Leb{n}$-summable function, and assume that $f$ is injective on the support of $g$. Then, we have
\begin{equation} \label{eq:change_variables_1}
\int_{\R^{n}} g(x) \, \J f(x) \, dx = \int_{f(\R^{n})} g(f^{-1}(y)) \, d \Haus{n}(y).
\end{equation}
Equivalently, if $h : \R^{m} \to \R$ is such that $h \circ f$ is $\Leb{n}$-summable and $f$ is injective on the support of $h$, then we have
\begin{equation} \label{eq:change_variables_2}
\int_{\R^{n}} h(f(x)) \, \J f(x) \, dx = \int_{f(\R^{n})} h(y) \, d \Haus{n}(y).
\end{equation}
If $g = \chi_{A}$ for some $\Leb{n}$-measurable set $A$, then
\begin{equation} \label{eq:area_formula_inj}
\Haus{n}(f(A)) = \int_{A} \J f(x) \, dx.
\end{equation}
\end{corollary}

\begin{remark}
Theorem \ref{change_variables_gen} and Corollary \ref{change_variables} hold also in the case $g : \R^{n} \to [0, + \infty]$ is $\Leb{n}$-measurable; however, the left hand sides of \eqref{eq:change_variables_gen} and \eqref{eq:change_variables_1} may be equal to $+ \infty$.
In addition, since any Borel function is Lebesgue measurable, Theorem \ref{change_variables_gen} and Corollary \ref{change_variables} are valid for all Borel functions $g : \R^{n} \to \R$ either nonnegative or $\Leb{n}$-summable.
\end{remark}

We list here some remarkable applications of the area formula.

\begin{example}[Length of a curve]
Let $n = 1, m \ge 1$. Assume $f : \R \to \R^{m}$ is Lipschitz and injective. It is clear that, for $\Leb{1}$-a.e. $t \in \R$,
\begin{equation*}
\J_{1} f(t) = |\dot{f}(t)|. 
\end{equation*}
Therefore, for any $a, b \in \R, a < b$, the length of a curve $C := f([a, b])$ is given by
\begin{equation*}
\Haus{1}(C) = \int_{a}^{b} |\dot{f}| \, dt,
\end{equation*}
thanks to \eqref{eq:area_formula_inj}.
\end{example}

\begin{example}[Surface area of a graph]
Let $n \ge 1$ and $m = n + 1$. Assume $g: \R^{n} \to \R$ is Lipschitz and define $f: \R^{n} \to \R^{n + 1}$ as $$f(x) := (x, g(x)).$$
Then
\[\nabla f(x) = \begin{pmatrix} 1 & \cdots & 0 & \\ \vdots & \ddots & \vdots \\ 
0 & \cdots & 1 & \\ \frac{\partial g}{\partial x_{1}} & \cdots & \frac{\partial g}{\partial x_{n}} & \end{pmatrix},\]
and so, by Cauchy-Binet formula (Proposition \ref{Cauchy_Binet}), we have
\begin{equation*}
\J f = \sqrt{1 + |\nabla g|^2}.
\end{equation*}
For any open set $\Omega \subset \R^{n}$, we define the graph of $g$ over $\Omega$ as
$$ \Gamma(g, \Omega) := \{ (x, g(x)) : x \in \Omega \} \subset \R^{n + 1}.$$
Therefore, \eqref{eq:area_formula_inj} yields
\begin{equation*}
\Haus{n}(\Gamma(g, \Omega)) = \int_{\Omega} \sqrt{1 + |\nabla g|^{2}} \, dx.
\end{equation*}
\end{example}

\begin{example}[Surface area of a parametric hypersurface]
Let $n \ge 1$ and $m = n + 1$. Assume $f : \R^{n} \to \R^{n + 1}, f = (f^1, \dots, f^{n + 1}),$ is Lipschitz and injective. For any $k \in \{1, \dots, n + 1 \}$, we define $$\hat{f}_{k} := (f^{1}, \dots, f^{k - 1}, f^{k + 1}, \dots, f^{n + 1});$$
that is, the vector valued map $f$ without its $k$-th component. Then, it is not difficult to see that $$\J \hat{f}_{k} = |\det{\nabla \hat{f}_{k}}|,$$
and so, as a consequence of Cauchy-Binet formula (Proposition \ref{Cauchy_Binet}), we have
\begin{equation*}
\J f = \sqrt{\sum_{k = 1}^{n + 1} (\J \hat{f}_{k})^{2}}.
\end{equation*}
Thus, if we define $\Sigma(f, \Omega) := f(\Omega)$, for any open set $\Omega \subset \R^{n}$ to be a portiong of the parametric hypersurface, \eqref{eq:area_formula_inj} yields
\begin{equation*}
\Haus{n}(\Sigma(f, \Omega)) = \int_{\Omega} \sqrt{\sum_{k = 1}^{n + 1} (\J \hat{f}_{k})^{2}} \, dx.
\end{equation*}
\end{example}

\subsection{The coarea formula}

In this subsection we assume $n \ge m$ and $f : \R^{n} \to \R^{m}$ to be Lipschitz.

\begin{lemma} Let $A$ be $\Leb{n}$-measurable. Then
\begin{enumerate}[i)]
\item $A \cap f^{-1}(y)$ is $\Haus{n - m}$ measurable for $\Leb{m}$-a.e. $y \in \R^{m}$.
\item the mapping $y \to \Haus{n - m}(A \cap f^{-1}(y))$ is $\Leb{m}$-measurable, and
\begin{equation*}
\int_{\R^{m}} \Haus{n - m}(A \cap f^{-1}(y)) \, d y \le c_{n, m} (\Lip(f))^{m} \Leb{n}(A).
\end{equation*}
\end{enumerate}
\end{lemma}

\begin{theorem}[Coarea formula] \label{coarea_formula}
Let $f : \R^{n} \to \R^{m}$ be Lipschitz and $n \ge m$. Then, for all $\Leb{n}$-measurable sets $A \subset \R^{n}$, we have
\begin{equation} \label{eq:coarea_formula}
\int_A \J f \, dx = \int_{\R^{m}} \Haus{n - m}(A \cap f^{-1}(y)) \, dy.
\end{equation}
\end{theorem}

Notice that the coarea formula can be seen as a generalized version of Fubini's theorem.

\begin{remark}[Morse-Sard theorem]
If we apply the coarea formula to $A = \{ \J f = 0 \}$, it is immediate to see that
\begin{equation*}
\Haus{n - m}( \{ \J f = 0 \} \cap f^{-1}(y)) = 0
\end{equation*}
for $\Leb{m}$-a.e. $y \in \R^{m}$. This is a weak variant of Morse-Sard theorem, which states that, if $f \in C^{k}(\R^{n}; \R^{m})$ for $k = 1 + n - m$, then
\begin{equation*}
\{\J f = 0 \} \cap f^{-1}(y) = \emptyset
\end{equation*}
for $\Leb{m}$-a.e. $y \in \R^{m}$.
\end{remark}

\begin{theorem}\label{integration_level_sets_gen}
Let $f : \R^{n} \to \R^{m}$ be Lipschitz and $n \ge m$. Then, for all $\Leb{n}$-summable functions $g : \R^{n} \to \R$, we have $g|_{f^{-1}(y)}$ is $\Haus{n - m}$-summable for $\Leb{m}$-a.e. $y \in \R^{m}$, and
\begin{equation} \label{eq:integration_level_sets_gen}
\int_{\R^{n}} g \, \J f \,dx = \int_{\R^{m}} \int_{f^{-1}(y)} g \, d \Haus{n - m} \, dy.
\end{equation}
\end{theorem}

\begin{remark}
Notice that $f^{-1}(y)$ is closed for all $y \in \R^{m}$, so that it is immediately $\Haus{n - m}$-measurable.
\end{remark}

We list now some relevant applications of the coarea formula.

\begin{theorem}[Polar coordinates] \label{polar_coordinates}
Let $g : \R^{n} \to \R$ be $\Leb{n}$-summable. Then
\begin{equation*} 
\int_{\R^{n}} g \, dx = \int_{0}^{+\infty} \int_{\partial B(0, \rho)} g \, d \Haus{n - 1} \, d \rho
\end{equation*}
In particular, for any $r > 0$ and $g$ such that $g \chi_{B(0, r)}$ is $\Leb{n}$-summable, we have
\begin{equation*}
\int_{B(0, r)} g \, dx = \int_{0}^{r} \int_{\partial B(0, \rho)} g \, d \Haus{n - 1} \, d \rho,
\end{equation*}
so that, for $\Leb{1}$-a.e. $r > 0$,
\begin{equation*}
\frac{d}{dr} \int_{B(0, r)} g \, dx = \int_{\partial B(0, r)} g \, d \Haus{n - 1}.
\end{equation*}
\end{theorem}
\begin{proof}
Apply Theorem \ref{integration_level_sets_gen} to $f(x) = |x|$, in the case $m = 1$. Then, for all $x \neq 0$, we have $$\nabla f(x) = \frac{x}{|x|}, \ \J f(x) = 1.$$
Finally, the second equality is a consequence of the first, as the third can be derived from the second.
\end{proof}

In general, in the case $m = 1$ it is easy to notice that $\J f = |\nabla f|$, so that we have the following result.

\begin{theorem}[Integration over level sets] \label{integration_level_sets}
Let $f : \R^{n} \to \R$ be Lipschitz. Then we have
\begin{equation} \label{eq:coarea_n_1}
\int_{\R^{n}} |\nabla f| \, dx = \int_{- \infty}^{+ \infty} \Haus{n - 1}(\{f = t\}) \, dt.
\end{equation}
If we assume also that ${\rm ess} \inf |\nabla f| > 0$ and we let $g : \R^{n} \to \R$ be $\Leb{n}$-summable, then for all $t \in \R$ we obtain
\begin{equation} \label{eq:integration_level_set}
\int_{\{f > t \}} g \, dx = \int_{t}^{+ \infty} \int_{\{f =s\}} \frac{g}{|\nabla f|} \, d \Haus{n - 1} \, ds.
\end{equation}
In particular, 
\begin{equation*}
\frac{d}{dt} \int_{\{f > t\}} g \, dx = - \int_{\{f = t\}} \frac{g}{|\nabla f|} \, d \Haus{n -1}
\end{equation*}
for $\Leb{1}$-a.e. $t \in \R$.
\end{theorem}
\begin{proof}
It is easy to see that \eqref{eq:coarea_n_1} is a simple consequence of \eqref{eq:coarea_formula} when $m = 1$. Then, by Theorem \ref{integration_level_sets_gen}, we have
\begin{align*}
\int_{\{f > t\}} g \, dx & = \int_{\R^{n}} \chi_{\{f > t\}} \frac{g}{|\nabla f|} \J f \, dx \\
& = \int_{- \infty}^{+ \infty} \int_{\{ f = s\}} \chi_{\{f > t\}} \frac{g}{|\nabla f|} \, d \Haus{n - 1} \, ds \\
& = \int_{t}^{+ \infty} \int_{\{ f = s\}} \chi_{\{f > t\}} \frac{g}{|\nabla f|} \, d \Haus{n - 1} \, ds.
\end{align*}
Then, the final equality follow easily by \eqref{eq:integration_level_set}.
\end{proof}

Finally, we conclude this section with the examination of the case in which $f$ is the distance function from a compact set.

\begin{theorem}[Integration over the level set of the distance function] \label{integration_distance_function}
Let $K\subset \R^{n}$ be a nonempty compact set and set $$d(x):= {\rm dist}(x, K).$$
Then, for all $0 < a < b$ and all $g : \R^{n} \to \R$ $\Leb{n}$-summable we have
\begin{equation*}
\int_{a}^{b} \int_{\{ d = t\}} g \, d \Haus{n - 1} \, dt = \int_{\{ a < d \le b\}} g \, dx.
\end{equation*}
In particular,
\begin{equation*}
\int_{a}^{b} \Haus{n - 1}(\{d = t\}) \, dt = \Leb{n}(\{ a < d \le b\}).
\end{equation*}
\end{theorem}
\begin{proof}
It is enough to prove that $|\nabla d(x)| = 1$ for $\Leb{n}$-a.e. $x \in \R^{n} \setminus K$, and then we need just to apply Theorem \ref{integration_level_sets}. We start by showing the $d$ is Lipschitz. Let $x \in \R^{n}$: there exists a $c \in K$ such that $|x - c| = d(x)$. By the triangle inequality, we have
\begin{equation*}
d(y) - d(x) \le |y - c| - |x - c| \le |x - y|.
\end{equation*}
If we interchange now the roles of $x$ and $y$, we see that we get 
\begin{equation*}
|d(y) - d(x)| \le |x - y|,
\end{equation*}
which shows that $d$ is Lipschitz with $\Lip(d) \le 1$. Hence, Rademacher's theorem implies that the function $d$ is differentiable in $\Leb{n}$-a.e. $x \in \R^n$. Let now $x \in \R^{n} \setminus K$ be such that $\nabla d(x)$ exists. Then we have $|\nabla d(x)| \le 1$. In addition, if we select $c \in K$ as above, we also have
\begin{equation*}
d(tx + (1 - t)c) = t |x - c|
\end{equation*}
for all $t \in [0, 1]$ (since the segment is the shortest path). Therefore, by taking a derivative in $t$, we get
\begin{equation*}
|x - c| = \nabla d(x) \cdot (x - c) \le |\nabla d(x)| |x - c|,
\end{equation*}
which immediately implies $|\nabla d(x)| \ge 1$ for $\Leb{n}$-a.e. $x \in \R^{n} \setminus K$. Thus, the proof is completed.
\end{proof}
