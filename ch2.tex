\section{Covering theorems and differentiation of measures}

In this section we introduce the coverings and differentiation theorems, fundamental tools of Geometric Measure Theory. The exposition is based mostly on \cite[Chapter 2]{AFP}.

\subsection{Covering theorems}
We say that a family of sets $\mathcal{F}$ is disjoint if $F \cap F' = \emptyset$ for
all $F,F' \in \mathcal{F}$, $F \neq F'$.
We notice that, since $\R^{n}$ is separable, every disjoint family of sets with nonempty interior is at most countable.

\begin{theorem}[Besicovitch covering theorem] \label{thm:Besicovitch_covering}
There exists a $\xi_n \in \N$ such that for all families of closed balls
$\mathcal{F}$ such that the set 
$A: = \{ x \in \R^n \mid \exists \varrho > 0 : \overline{B(x,\varrho)} \in
\mathcal{F}\}$ is bounded, there exists at most $\xi_n$ disjoint subfamilies
$\mathcal{F}_i \subset \mathcal{F}$ such that 
\[
A \subset \bigcup_{i=1}^{\xi_n} \bigcup_{\overline{B} \in \mathcal{F}_i}
\overline{B}.
\]
\end{theorem}

\begin{remark} \label{rem:open_balls}
The balls in the statement of Theorem \ref{thm:Besicovitch_covering} may be taken to be open.
\end{remark}

\begin{theorem}[Consequence of Besicovitch theorem] \label{thm:Besicovitch_covering_1}
Let $A$ be a bounded set and $\varrho : A \to (0,\infty)$. Then there exists $S
\subset A$ at most countable such that 
\(
A \subset \bigcup_{x\in S} B(x,\varrho(x)) 
\)
and such that every point of $\R^{n}$ belongs to at most $\xi_{n}$ open balls centered in points of $S$; that is,
\[
\sum_{x \in S} \chi_{B(x,\varrho(x))}(y) \leq \xi_n
\]
for all $y \in \R^n$, where $\xi_{n}$ is the same constant of Theorem \ref{thm:Besicovitch_covering}.
\end{theorem}
\begin{proof}
Let $\mathcal{F} := \{B(x,\varrho(x)) \mid x \in A\}$ and apply Besicovitch covering theorem (Theorem \ref{thm:Besicovitch_covering}): there exists $\xi_{n} \in \N$ and $\mathcal{F}_{i}, \dots, \mathcal{F}_{\xi_{n}}$ disjoint families of open balls (thanks to Remark \ref{rem:open_balls}) such that
\begin{equation*}
A \subset \bigcup_{i = 1}^{\xi_{n}} \bigcup_{B \in \mathcal{F}_{i}} B.
\end{equation*} 
Then, we set $S$ to be the set of centers of the balls in the families $\mathcal{F}_{i}$, for $i \in \{1, \dots, \xi_{n}\}$. Clearly, each one of these families is at most countable, being disjoint, so that $S$ is at most countable. This ends the proof.
\end{proof}

We employ now these results to show a covering theorem for Radon measure due to Vitali, which is the general version of Lemma \ref{lemmaVitaliCovering}. To this purpose we define the notion of fine covering.

\begin{definition}
Let $\mathcal{F}$ be a family of closed balls and $A \subset \R^n$.
$\mathcal{F}$ is called a {\em fine covering} of $A$ if 
\[
\inf \left\{ \varrho > 0 \mid \overline{B(x,\varrho)} \in \mathcal{F}\right \} =
0 \quad \text{for all $x\in A$}.
\]
\end{definition}

\begin{theorem}[Vitali covering theorem] \label{thm:Vitali_covering_theorem}
Let $A$ be a bounded Borel set and $\mathcal{F}$ be a fine covering of $A$. For any $\mu \in \mathcal{M}^+_{\rm loc}(\R^n)$ there exists $\mathcal{F}' \subset
\mathcal{F}$ disjoint such that 
\[
\mu \left( A \setminus \bigcup_{\overline{B}
\in \mathcal{F}'} \overline{B} \right) = 0
\]
\end{theorem}
\begin{proof}
Let $\xi = \xi_{n}$ be the constant from Besicovitch covering theorem (Theorem \ref{thm:Besicovitch_covering}), and let $$\delta := 1 - \frac{1}{2 \xi}.$$
It is clear that $\delta \in (0, 1)$.

{\em Claim}: there exists a finite subfamily $\mathcal{F}_{1} \subset \mathcal{F}$ such that
\begin{equation} \label{eq:claim_one_Vitali}
\mu \left ( A \cap \bigcup_{ \overline{B} \in \mathcal{F}_{1}} \overline{B} \right ) \ge \frac{1}{2 \xi} \mu(A).
\end{equation}
In order to prove this claim, we notice that, since $\mathcal{F}$ is a covering of $A$, there exist disjoint at most countable families $\mathcal{D}_{1}, \dots, \mathcal{D}_{\xi}$ which cover $A$. In particular, there exists $i \in \{1, \dots, \xi \}$ such that 
\begin{equation*}
\mu \left ( A \cap \bigcup_{ \overline{B} \in \mathcal{D}_{i}} \overline{B} \right ) \ge \frac{1}{\xi} \mu(A).
\end{equation*}
The reason of this lies in the fact that
\begin{equation*}
\mu(A) = \mu \left ( A \cap \bigcup_{j = 1}^{\xi} \bigcup_{ \overline{B} \in \mathcal{D}_{j}}  \overline{B} \right) \le \sum_{j = 1}^{\xi} \mu \left ( A \cap \bigcup_{ \overline{B} \in \mathcal{D}_{j}}  \overline{B} \right ).
\end{equation*}
In particular, we can select a finite subfamily $\mathcal{F}_{1} \subset \mathcal{D}_{i}$ such that \eqref{eq:claim_one_Vitali} holds.

We let now $$A_{1} := A \setminus \bigcup_{ \overline{B} \in \mathcal{F}_{1}}  \overline{B},$$
and we apply the same procedure to the family 
\begin{equation*}
\tildef{\mathcal{F}} := \left \{ \overline{B(x, \rho)} \in \mathcal{F} :  \overline{B(x, \rho)} \cap \bigcup_{ \overline{B} \in \mathcal{F}_1}  \overline{B} = \emptyset \right \}.
\end{equation*}
Notice that $\tildef{\mathcal{F}}$ covers $A_{1}$, since $\mathcal{F}$ is a fine cover of $A$. Indeed, if $x \in A_{1}$, then there exists $\rho > 0$ such that $\overline{B(x, \rho)} \cap \bigcup_{ \overline{B} \in \mathcal{F}_1}  \overline{B} = \emptyset$, being $\mathcal{F}_1$ finite and the balls closed.
In this way, we find a disjoint subfamily $\mathcal{F}_{2} \subset \mathcal{F}$ disjoint from $\mathcal{F}_{1}$ such that
\begin{equation*}
\mu \left ( A_{1} \cap \bigcup_{ \overline{B} \in \mathcal{F}_{2}}  \overline{B} \right ) \ge \frac{1}{2 \xi} \mu(A_{1}).
\end{equation*}
In addition, we observe that, by \eqref{eq:claim_one_Vitali},
\begin{equation} \label{eq:delta_contraction}
\mu(A_{1}) = \mu \left ( A \setminus \bigcup_{ \overline{B} \in \mathcal{F}_{1}}  \overline{B} \right ) = \mu(A) - \mu \left ( A \cup \bigcup_{ \overline{B} \in \mathcal{F}_{1}}  \overline{B} \right ) \le \mu(A) - \frac{1}{2 \xi} \mu(A)  = \delta \mu(A).
\end{equation}
By iteration, we construct a collection of finite disjoint family of closed balls $\{\mathcal{F}_{k}\}_{k \in \N}$ and a decreasing family of sets $\{A_{k}\}_{k \in \N}$, by setting $A_{0} = A$ and
\begin{equation*} 
A_{k} = A_{k - 1} \setminus \bigcup_{ \overline{B} \in \mathcal{F}_{k}}  \overline{B}.
\end{equation*}
By \eqref{eq:delta_contraction}, it is easy to see that
\begin{equation*}
\mu(A_{k + 1}) \le \delta \mu(A_{k}),
\end{equation*}
which easily implies
\begin{equation} \label{eq:delta_contraction_1}
\mu(A_{k}) \le \delta^{k} \mu(A).
\end{equation}
If we let $$\mathcal{F}' := \bigcup_{k = 1}^{\infty} \mathcal{F}_{k},$$
it is easy to see that $\mathcal{F}'$ is countable and disjoint, and satisfies
\begin{equation*}
A \setminus \bigcup_{ \overline{B} \in \mathcal{F}'}  \overline{B} \subset \bigcap_{k = 1}^{\infty} A_{k}.
\end{equation*} 
Thus, \eqref{eq:delta_contraction_1} implies
\begin{equation*}
\mu \left ( A \setminus \bigcup_{ \overline{B} \in \mathcal{F}'} \overline{B} \right ) \le \lim_{k \to + \infty} \mu(A_{k}) \le \mu(A) \lim_{k \to + \infty} \delta^{k} = 0,
\end{equation*}
and this ends the proof.
\end{proof}

\subsection{Differentiation of Radon and Hausdorff measures}

One of the main applications of the covering theorems from the previous section is the differentiation of measures, which allows us to pass from local properties to global ones.

\begin{theorem}[Lebesgue-Besicovitch differentiation theorem] \label{thm:Leb_Bes_diff}
Let $\mu \in \mathcal{M}^+_{\rm loc}(\Omega)$, $\lambda \in
\mathcal{M}_{\rm loc}(\Omega;\R^m)$, $\lambda \ll \mu$. Then, for $\mu$-a.e. $x \in
\supp \mu$, the limit 
\begin{equation} \label{eq:limit_measure_der}
D_{\mu}\lambda(x) := \lim_{\varrho \to 0} \frac{\lambda(B(x,\varrho))}{\mu(B(x,\varrho))}
\end{equation}
exists in $\R^m$. In addition, we have $\lambda = D_{\mu}\lambda \, \mu$.
\end{theorem}

The limit in \eqref{eq:limit_measure_der}, sometimes also denoted by $(d\lambda/d\mu)(x)$, is called the {\em derivative}, or the {\em density}, of $\lambda$ with respect to $\mu$. 

\begin{corollary}\label{cor:Leb_diff}
Let $\mu \in \mathcal{M}^+_{\rm loc}(\Omega)$ and $f \in L^{1}_{\rm loc}(\Omega, \mu; \R^{m})$. Then we have
\begin{equation*}
f(x) = \lim_{\rho \to 0} \frac{1}{\mu(B(x,\rho))}\int_{B(x, \rho)} f(y) \, d \mu(y)
\end{equation*} 
for $\mu$-a.e. $x \in \Omega$.
\end{corollary}
\begin{proof}
Let $\lambda := f \mu$. It is easy to check that $\lambda \in \mathcal{M}_{\rm loc}(\Omega; \R^{m})$.
By Theorem \ref{thm:Leb_Bes_diff}, we know that, for $\mu$-a.e. $x \in \Omega$, there exists the limit 
\begin{equation*}
D_{\mu}\lambda(x) := \lim_{\rho \to 0} \frac{\lambda(B(x,\rho))}{\mu(B(x,\rho))} = \lim_{\rho \to 0} \frac{1}{\mu(B(x,\rho))}\int_{B(x, \rho)} f(y) \, d \mu(y),
\end{equation*}
and it satisfies $\lambda = D_{\mu}\lambda \, \mu$. Thus, we have $(f - D_{\mu}\lambda) \, \mu = 0$, which implies $f(x) = D_{\mu}\lambda(x)$ for $\mu$-a.e. $x \in \Omega$.
\end{proof}

\begin{remark}
If we apply Corollary \ref{cor:Leb_diff} to the case $\mu = \Leb{n}$ and $f \in L^{1}_{\rm loc}(\Omega; \R^{m})$, we obtain the classical version of Lebesgue's differentiation theorem:
\begin{equation*}
f(x) = \lim_{\rho \to 0} \meanint_{B(x, \rho)} f(y) \, dy = \lim_{\rho \to 0} \frac{1}{\omega_{n} \rho^{n}} \int_{B(x, \rho)} f(y) \, dy
\end{equation*}
for $\Leb{n}$-a.e. $x \in \Omega$.
In particular, if $f = \chi_{E}$ for some Lebesgue measurable set $E$ in $\R^{n}$, we obtain
\begin{equation} \label{density_points_set}
\chi_{E}(x) = \lim_{\rho \to 0} \frac{|E \cap B(x, \rho)|}{|B(x, \rho)|} =  \lim_{\rho \to 0} \frac{|E \cap B(x, \rho)|}{\omega_{n} \rho^{n}}
\end{equation}
for $\Leb{n}$-a.e. $x \in \R^{n}$.
\end{remark}

Actually, it is not difficult to show that the statement of Corollary \ref{cor:Leb_diff} may be refined to a slightly stronger version.

\begin{corollary}
Let $\mu \in \mathcal{M}^{+}_{\rm loc}(\Omega)$ and $f \in L^{1}(\Omega, \mu; \R^{m})$. Prove that for $\mu$-a.e. $x \in \Omega$ we have
\begin{equation} \label{eq:Lebesgue_points}
\lim_{\rho \to 0} \frac{1}{\mu(B(x, \rho))} \int_{B(x, \rho)}|f(y) - f(x)| \, d \mu(y) = 0.
\end{equation}
\end{corollary}
Any point $x \in \Omega$ for which \eqref{eq:Lebesgue_points} holds is called {\em Lebesgue point} (or {\em approximate continuity point}) of $f$.

\begin{proof}[Hint of the proof]
It is enough to apply Corollary \ref{cor:Leb_diff} to the measures $\nu_{q} := |f - q| \mu$, for $q \in \Q$, and then to exploit the fact that $\Q$ is countable and dense in $\R$.
\end{proof}

We recall that the Hausdorff measures $\Haus{\alpha}$ for $\alpha \in [0, n)$ are Borel regular, but not Radon (see Remark \ref{rem:Haus_not_Radon}). Hence, the Lebesgue-Besicovitch differentiation theorem (Theorem \ref{thm:Leb_Bes_diff}) does not apply to such measures. Nevertheless, it is possible to define and to study a notion of density with respect to the Hausdorff measures. Such analysis is relevant in many applications, since it is useful to compare a generic Radon measure $\mu$ to $\Haus{\alpha}$ for some $\alpha \in [0, n]$, in order to to have some idea of the ``dimensionality'' of $\mu$.

\begin{definition} \label{def:alpha_dimensional_densities}
Let $\mu \in \mathcal{M}^{+}_{\rm loc}(\Omega)$ and $\alpha \in [0, n]$. For all $x \in \Omega$, we define the {\em upper and lower $\alpha$-dimensional densities} of $\mu$ at $x$ as
\begin{equation*}
\Theta^{*}_{\alpha}(\mu, x) := \limsup_{\rho \to 0} \frac{\mu(B(x, \rho))}{\omega_{\alpha} \rho^{\alpha}} \quad \text{and} \quad \Theta_{* \alpha}(\mu, x) := \liminf_{\rho \to 0} \frac{\mu(B(x, \rho))}{\omega_{\alpha} \rho^{\alpha}}.
\end{equation*}
If $\Theta^{*}_{\alpha}(\mu, x) = \Theta_{* \alpha}(\mu, x)$, the common value is denoted by $\Theta_{\alpha}(\mu, x)$.
In the case $\mu = \Haus{\alpha} \res E$, for a $\Haus{\alpha}$-measurable set $E$ such that $\Haus{\alpha}(E) < + \infty$, for simplicity we write
\begin{equation*}
\Theta^{*}_{\alpha}(E, x) := \Theta^{*}_{\alpha}(\Haus{\alpha} \res E, x), \ \Theta_{* \alpha}(E, x) := \Theta_{* \alpha}(\Haus{\alpha} \res E, x) \text{ and }  \Theta_{\alpha}(E, x) := \Theta_{\alpha}(\Haus{\alpha} \res E, x).
\end{equation*}
\end{definition}

\begin{remark} \label{rem:Borel_densities}
Recall that, by Proposition \ref{prop:Haus_Radon_res}, $\mu = \Haus{\alpha} \res E$ is a Radon measure, for any $\Haus{\alpha}$-measurable set $E$ such that $\Haus{\alpha}(E) < + \infty$, so that the definitions of $\Theta^{*}_{\alpha}(E, x), \Theta_{* \alpha}(E, x)$ and $\Theta_{\alpha}(E, x)$ are well posed.

In addition, for any $\mu \in \mathcal{M}^{+}_{\rm loc}(\Omega)$ and $\alpha \in [0, n]$, the functions $\Theta^{*}_{\alpha}(\mu, \cdot)$ and $\Theta_{* \alpha}(\mu, \cdot)$ are Borel, since the map $x \to \mu(B(x, \rho))$ is lower semicontinuous, for any fixed $\rho > 0$, by Fatou's lemma.
\end{remark}

\begin{theorem}[$\alpha$-dimensional densities of Radon measures] \label{thm:alpha_density}
Let $\mu \in \mathcal{M}^{+}_{\rm loc}(\Omega)$ and $\alpha \in [0, n]$. Then, for all $B \in \mathcal{B}(\Omega)$ and $t > 0$, the following implications hold:
\begin{align}
\Theta^{*}_{\alpha}(\mu, x) \ge t \ \forall x \in B & \Longrightarrow \ \mu \ge t \Haus{\alpha} \res B, \label{eq:lower_bound_density} \\
\Theta^{*}_{\alpha}(\mu, x) \le t \ \forall x \in B & \Longrightarrow \ \mu \le 2^{\alpha} t \Haus{\alpha} \res B. \label{eq:upper_bound_density}
\end{align}
\end{theorem}
\begin{proof}
Without loss of generality, let $t = 1$ and $B$ bounded. Indeed, we may set $\tildef{\mu} := t \mu$; and, if $B$ is unbounded, we can cover it with countably many bounded sets.

\smallskip

We start by proving \eqref{eq:lower_bound_density}. To this purpose, let $B' \in \mathcal{B}(\Omega)$ be a subset of $B$, $\delta \in (0, 1)$, and $A \supset B'$ be an open bounded set in $\Omega$. Let us consider the open balls $B(x_{j}, r_{j})$ such that $B(x_{j}, r_{j}) \subset A, 2 r_{j} < \delta$ and $$\mu(B(x_{j}, r_{j})) \ge (1 - \delta) \omega_{\alpha} r_{j}^{\alpha}.$$
By Theorem \ref{thm:Besicovitch_covering_1}, there exists a subfamily $\{B(x_{j_{k}}, r_{j_{k}})\}_{k \in \N}$ such that
\begin{equation*}
\sum_{k \in \N} \chi_{B(x_{j_{k}}, r_{j_{k}})} \le \xi_{n}
\end{equation*}
and $B' \subset \bigcup_{k \in \N} B(x_{j_{k}}, r_{j_{k}})$. Then, we have
\begin{equation*}
\Haus{\alpha}_{\delta}(B') \le \sum_{k \in \N} \omega_{\alpha} r_{j_{k}}^{\alpha} \le \frac{1}{1 - \delta} \sum_{k \in \N} \mu(B(x_{j_{k}}, r_{j_{k}})) \le \frac{\xi_{n}}{1 - \delta} \mu(A),
\end{equation*}
which easily implies
\begin{equation*}
\Haus{\alpha}(B') \le \xi_{n} \mu(A) < \infty.
\end{equation*}
This means that $\Haus{\alpha} \res B'$ is a Radon measure, thanks to Proposition \ref{prop:Haus_Radon_res}. Therefore, we can apply Vitali covering theorem (Theorem \ref{thm:Vitali_covering_theorem}) to $\Haus{\alpha} \res B'$ and to the covering of closed balls $C$ in $A$ with diameter less than $\delta$ and such that
\begin{equation*}
\mu(C) \ge (1 - \delta) \omega_{\alpha} \frac{({\rm diam} (C))^{\alpha}}{2^{\alpha}}.
\end{equation*}
Hence, we get a disjoint subfamily $\{C_{i}\}$, for which we have
%\begin{equation*}
%\left ( \Haus{\alpha} \res B' \right)\left ( A \setminus \bigcup_{i} C_{i} \right ) = \Haus{\alpha} \left ( B' \setminus \bigcup_{i} C_{i} \right ) = 0,
%\end{equation*}
\begin{equation*}
\Haus{\alpha}_{\delta}(B') \le \sum_{i} \frac{\omega_{\alpha}}{2^{\alpha}} ({\rm diam} (C_{i}))^{\alpha} \le \frac{1}{1 - \delta} \sum_{i} \mu(C_{i}) \le \frac{\mu(A)}{1 - \delta}.
\end{equation*}
Thus, we obtain $\Haus{\alpha}(B') \le \mu(B')$ for any Borel set $B' \subset B$, since $A$ and $\delta$ are arbitrary and $\mu$ is outer regular (see Theorem \ref{thm:regularity_Radon}). This shows \eqref{eq:lower_bound_density}.

\smallskip

We pass now to the proof of \eqref{eq:upper_bound_density}. Let $B' \in \mathcal{B}(\Omega)$ be a subset of $B$, and $\tau > 1$. For any $k \in \N, k \ge 1$ we set
\begin{equation*}
B_{k} := \left \{ x \in B' : \frac{\mu(B(x, \rho))}{\omega_{\alpha} \rho^{\alpha}} < \tau  \ \forall \rho \in \left (0, \frac{1}{k} \right ) \right \}. 
\end{equation*}
Notice that $B_{k} \neq \emptyset$, for $k$ large enough, since
\begin{equation*}
\limsup_{\rho \to 0} \frac{\mu(B(x, \rho))}{\omega_{\alpha} \rho^{\alpha}} \le 1 
\end{equation*}
by our assumption. In addition, $B_{k} \subset B_{k + 1}$ and $\bigcup_{k = 1}^{\infty} B_{k} = B'$. For any $k$, let $\{C_{i, k}\}$ be a covering of $B_{k}$ of sets with diameter less than $1/k$ such that
\begin{equation} \label{eq:estimate_C_i_k_density}
\sum_{i} \omega_{\alpha} \left ( \frac{{\rm diam}(C_{i, k})}{2} \right )^{\alpha} < \Haus{\alpha}_{\frac{1}{k}}(B_{k}) + \frac{1}{k}.
\end{equation}
Without loss of generality, assume $x_{i} \in B_{k} \cap C_{i, k}$ and let $C_{i, k}' = B(x_{i}, 2 \rho_{i, k})$, where $$\rho_{i, k} := \frac{\diam(C_{i, k})}{2}.$$
Therefore, $B_{k} \subset \bigcup_{i} C_{i, k}'$ and, by \eqref{eq:estimate_C_i_k_density}, we have 
\begin{equation*}
\mu(B_{k}) \le \sum_{i} \mu(C_{i, k}') \le \tau \sum_{i} \omega_{\alpha} (2 \rho_{i, k})^{\alpha} < \tau \left ( 2^{\alpha} \Haus{\alpha}(B') + \frac{1}{k} \right ).
\end{equation*}
Then we send $k \to + \infty$ and $\tau \to 1^{+}$, and this ends the proof.
\end{proof}

We illustrate now two useful consequences of Theorem \ref{thm:alpha_density}.

\begin{corollary} \label{cor:Hausdorff_density_estimate}
Let $\mu \in \mathcal{M}^{+}_{\rm loc}(\Omega)$ and $\alpha \in [0, n]$. Then we have the following statements:
\begin{enumerate}
\item $\Theta^{*}_{\alpha}(\mu, x) < \infty$ for $\Haus{\alpha}$-a.e. $x \in \Omega$,
\item if $\mu(B) = 0$ for some $B \in \mathcal{B}(\Omega)$, then $\Theta_{\alpha}(\mu, x) = 0$ for $\Haus{\alpha}$-a.e. $x \in B$.
\end{enumerate}
\end{corollary}
\begin{proof}
We start by showing point 1. Let $D \Subset \Omega$ be a Borel set satisfying
\begin{equation*} 
D \subset D_{\infty} := \{ x \in \Omega : \Theta^{*}_{\alpha}(\mu, x) = \infty \}.
\end{equation*}
Then we have $\Theta^{*}_{\alpha}(\mu, x) \ge t$ for any $x \in D$ and $t > 0$, so that, by \eqref{eq:lower_bound_density}, we obtain $$\mu(D) \ge t \Haus{\alpha}(D)$$ for any $t > 0$. Since $\mu(D) < \infty$, this implies $\Haus{\alpha}(D) = 0$.
Then, for any $k \in \N, \ge 1$, let 
\begin{equation*}
D_{k} := \left \{ x \in D_{\infty}, |x| \le k, \dist(x, \partial \Omega) \ge \frac{1}{k} \right \}.
\end{equation*}
Recall that $\Theta^{*}_{\alpha}(\mu, \cdot)$ is a Borel function (see Remark \ref{rem:Borel_densities}), so that $D_{\infty}$ and $D_{k}$ are Borel sets.
It is plain to see that $D_{k} \subset D_{\infty}$, $D_{k} \Subset \Omega$ and $\bigcup_{k = 1}^{\infty} D_{k} = D_{\infty}$.
Thus, we have
\begin{equation*}
\Haus{\alpha}( D_{\infty}) \le \sum_{k = 1}^{\infty} \Haus{\alpha}(D_{k}) = 0.
\end{equation*}
\smallskip

As for point 2, let $t > 0$, $B \in \mathcal{B}(\Omega)$ be such that $\mu(B) = 0$, and set
\begin{equation*}
E_{t} := \left \{ x \in B : \Theta^{*}_{\alpha}(\mu, x) > t \right \}.
\end{equation*}
Then, by \eqref{eq:lower_bound_density}, we have
\begin{equation*}
0 = \mu(B) \ge \mu(E_{t}) \ge t \Haus{\alpha}(E_{t}).
\end{equation*}
Since clearly $E_{0} = \bigcup_{k =1}^{\infty} E_{\frac{1}{k}}$, we have 
\begin{equation*}
\Haus{\alpha}(E_{0}) \le \sum_{k = 1}^{\infty} \Haus{\alpha}(E_{\frac{1}{k}}) = 0,
\end{equation*}
and this ends the proof.
\end{proof}

Let us consider the case $\mu = \Haus{\alpha} \res E$, for some $\Haus{\alpha}$-measurable set $E$ in $\R^{n}$ such that $\Haus{\alpha}(E) < + \infty$. We start by considering the extreme values of $\alpha$:
\begin{itemize}
\item if $\alpha = n$, thanks to \eqref{density_points_set}, we know that $\Theta_{n}(E, x) = \chi_{E}(x)$ for $\Leb{n}$-a.e. $x \in \R^{n}$; 
\item if $\alpha = 0$, it is easy to check that
\begin{equation*}
\Theta_{0}(E, x) = \lim_{\rho \to 0} \Haus{0}(E \cap B(x, \rho)) = \begin{cases} 1 & \text{if} \ x \in E \\
0 & \text{if} \ x \notin E
\end{cases} = \chi_{E}(x),
\end{equation*} 
since $\Haus{0}(E) < \infty$ implies that $E$ is finite and therefore discrete.
\end{itemize}
This means that, for $\alpha \in \{0, n\}$, we have $\Theta_{\alpha}(E, x) = \chi_{E}(x)$ for $\Haus{\alpha}$-a.e. $x \in \R^{n}$.
Instead, in the case $\alpha \in (0, n)$, we cannot hope for such a strong result, in general. However, we are able to obtain the following estimates.

\begin{proposition}\label{prop:Hausdorff_density_estimate_1}
Let $\alpha \in [0, n]$ and $E$ be an $\Haus{\alpha}$-measurable set in $\R^{n}$ such that $\Haus{\alpha}(E) < \infty$. Then we have
\begin{enumerate}
\item $\Theta_{\alpha}(E, x) = 0$ for $\Haus{\alpha}$-a.e. $x \notin E$,
\item $2^{-\alpha} \le \Theta^{*}_{\alpha}(E, x) \le 1$ for $\Haus{\alpha}$-a.e. $x \in E$.
\end{enumerate}
\end{proposition}
\begin{proof}
Point 1 follows immediately by applying Corollary \ref{cor:Hausdorff_density_estimate} to the Radon measure $\Haus{\alpha} \res E$, since clearly $(\Haus{\alpha} \res E)(\R^{n} \setminus E) = 0$.
\smallskip

In order to prove point 2, we set 
\begin{equation*}
B_{t} := \{ x \in E :  \Theta^{*}_{\alpha}(E, x) \le t \}
\end{equation*}
for $t < 2^{- \alpha}$. By \eqref{eq:upper_bound_density} to the Radon measure $\Haus{\alpha} \res E$, we have
\begin{equation*}
\Haus{\alpha}(B_{t}) = (\Haus{\alpha} \res E)(B_t) \le 2^{\alpha} t \Haus{\alpha}(B_{t}),
\end{equation*}
which implies $\Haus{\alpha}(B_{t}) = 0$. On the other hand, if we set
\begin{equation*}
C_{t} := \{ x \in E : \Theta^{*}_{\alpha}(E, x) \ge t \}
\end{equation*}
for $t > 1$, and we apply \eqref{eq:lower_bound_density} to $\Haus{\alpha} \res E$, we obtain
\begin{equation*}
\Haus{\alpha}(C_{t}) \ge t \Haus{\alpha}(C_{t}),
\end{equation*}
which yields $\Haus{\alpha}(C_{t}) = 0$.
\end{proof}

We notice that we do not have any general result on lower bounds for $\Theta_{* \alpha}(E, x)$, and this is the reason why we cannot ensure the existence of the full limit in general. However, as we shall see in the following, in the case $\alpha = k \in \{1, \dots, n - 1\}$ there is a way to characterize the sets for which $\Theta_{k}(E, x)$ is well defined and equal to $1$ for $\Haus{k}$-a.e. $x \in E$.

\section{Fine properties of Lipschitz functions}

We devote this section to the discussion of some properties of Lipschitz functions, which proved to be very useful in the framework of Geometric Measure Theory. 
The choice of working with Lipschitz functions is due to the fact that such functions have a less rigid structure than $C^{1}$-differentiable functions (for instance, extension theorems are much easier to prove, see McShane's lemma), while they enjoy differentiability properties almost everywhere (see Rademacher's theorem).

\begin{lemma}[McShane's lemma] \label{McShane_lemma}
Let $E \subset \R^{n}$ and $f: E \to \R$ be a Lipschitz function. Then the function $f^{+} : \R^{n} \to \R$ defined as
\begin{equation*}
f^{+}(x) := \inf \{ f(y) + \Lip(f, E) |x - y| : y \in E\}
\end{equation*}
is Lipschitz and it satisfies $f^{+}(x) = f(x)$ for all $x \in E$ and $\Lip(f, E) = \Lip(f^{+}, \R^{n})$.
\end{lemma}
\begin{proof}
For any $x, z \in \R^{n}$, by the triangle inequality, we have
\begin{equation*}
f^{+}(x) \le \inf \{ f(y) + \Lip(f, E) (|x - z| + |z - y|) : y \in E \} = f^{+}(z) + \Lip(f, E) |x - z|.
\end{equation*}
Then, interchanging the role of $x$ and $z$, we immediately get
\begin{equation*}
|f^{+}(x) - f^{+}(z)| \le \Lip(f, E) |x - z|,
\end{equation*}
which implies $\Lip(f^{+}, \R^{n}) \le \Lip(f, E)$.
Now, let $x \in E$. It is easy to see that $f^{+}(x) \le f(x)$. In order to obtain the reverse inequality, notice that 
\begin{equation*}
f(x) \le f(y) + \Lip(f, E)|x - y|
\end{equation*}
for any $y \in E$, since $f$ is Lipschitz on $E$. By taking the infimum in $y \in E$, we get $f(x) \le f^{+}(x)$, so that $f^{+}(x) = f(x)$ for all $x \in E$. Finally, this identity implies
\begin{equation*}
\Lip(f, E) = \Lip(f^{+}, E) \le \Lip(f^{+}, \R^{n}),
\end{equation*}
from which we conclude that $\Lip(f, E) = \Lip(f^{+}, \R^{n})$.
\end{proof}

\begin{remark} The extension given in McShane's lemma is the largest Lipschitz extension of $f$ preserving the Lipschitz constant. Indeed, let $\tilde{f}$ be another such extension: then, for any $x \in \R^{n}$ and $y \in E$, we have
\begin{equation*}
\tilde{f}(x) \le f(y) + \Lip(E, f) |y - x|,
\end{equation*} 
so that, by taking the infimum in $y \in E$, we obtain $\tilde{f}(x) \le f^{+}(x)$. On the other hand, arguing analogously, one can show that the smaller extension is given by
\begin{equation*}
f^{-}(x) := \sup \{ f(y) - \Lip(f, E) |x - y| : y \in E \}.
\end{equation*}
\end{remark}

It is not difficult to see that McShane's lemma can be extended to vector valued Lipschitz functions by hands; however, in such a way we loose the equality between the Lipschitz constants.

\begin{corollary}
Let $E \subset \R^{n}$ and $f: E \to \R^{m}$ be a Lipschitz function. Then there exists a Lipschitz function $\tildef{f} : \R^{n} \to \R^{m}$ such that $\tildef{f} = f$ on $E$ and $\Lip(\tildef{f}, \R^{n}) \le \sqrt{m} \Lip(f, E)$.
\end{corollary}
\begin{proof}
Apply McShane's lemma (Lemma \ref{McShane_lemma}) to each component of $f$, thus defining
\begin{equation*}
\tildef{f} := (f_{1}^{+}, \dots, f_{m}^{+}).
\end{equation*}
Then it is easy to see that $\tildef{f}= f$ on $E$. As for the Lipschitz constant, notice that
\begin{equation*}
|\tildef{f}(x) - \tildef{f}(y)|^{2} = \sum_{i = 1}^{m}|f_{i}^{+}(x) - f_{i}^{+}(y)|^{2} \le m (\Lip(f, E))^{2} |x - y|^2.
\end{equation*}
This ends the proof.
\end{proof}

A more refined result was found by M. D. Kirszbraun (\cite[2.10.43]{Fe} and \cite[Theorem I.7.2]{maggi2012sets}).

\begin{theorem}[Kirszbraun theorem]
Let $E \subset \R^{n}$ and $f: E \to \R^{m}$ be a Lipschitz function. Then there exists a Lipschitz function $g: \R^{n} \to \R^{m}$ such that $g = f$ on $E$ and $\Lip(g, \R^{n}) = \Lip(f, E)$.
\end{theorem}

A practical consequence of these extension results for Lipschitz functions is that we may always assume, without loss of generality, that our Lipschitz maps are defined on the whole space $\R^{n}$.

We shall now see that, quite surprisingly, the Lipschitz continuity property is enough to ensure differentiability outside of a Lebesgue negligible set. We start by recalling the notion of differentiability.

\begin{definition}
A function $f : \R^{n} \to \R^{m}$ is {\em differentiable} at $x \in \R^{n}$ if there exists a linear mapping $L: \R^{n} \to \R^{m}$ such that
\begin{equation*}
\lim_{y \to x} \frac{|f(y) - f(x) - L(y - x)|}{|y - x|} = 0.
\end{equation*}
This linear mapping is denoted by $\nabla f(x)$ or $d f(x)$.
\end{definition}

\begin{theorem}[Rademacher's theorem] \label{thm:Rademacher}
Let $f : \R^{n} \to \R^{m}$ be a locally Lipschitz function. Then $f$ is differentiable at $\Leb{n}$-a.e. $x \in \R^{n}$. In particular, $\nabla f(x)$ is well defined for $\Leb{n}$-a.e. $x \in \R^{n}$ and belongs to $L^{\infty}_{\rm loc}(\R^{n}; \R^{m} \times \R^{n})$, with $$\|\nabla f\|_{L^{\infty}(K; \R^{m} \times \R^{n})} \le \Lip(f, K)$$ for any compact set $K$.
\end{theorem}

An interesting consequence of this result is that the differential of a Lipschitz function vanishes on the level sets of the function.

\begin{theorem}
Let $f: \R^{n} \to \R^{m}$ be locally Lipschitz and $t \in \R$. Then $\nabla f(x) = 0$ for $\Leb{n}$-a.e. $x \in \{ f = t\} := \{ y \in \R^{n}: f(y) = t\}$.
\end{theorem}

\section{The area and Gauss--Green formulas}


\subsection{Linear maps and Jacobians}

We recall here some standard definitions and facts from linear algebra.

\begin{definition} \hfill
\begin{enumerate}[i)]
\item A linear map $O : \R^{n} \to \R^{m}$ is {\em orthogonal} if $$(Ox) \cdot (Oy) = x \cdot y$$ for all $x, y \in \R^{n}$.
\item A linear map $S: \R^{n} \to \R^{m}$ is {\em symmetric} if $$x \cdot (Sy) = (Sx) \cdot y$$ for all $x, y \in \R^{n}$.
\item Let $A: \R^{n} \to \R^{m}$. The {\em adjoint} of $A$ is the linear map $A^{*}: \R^{m} \to \R^{n}$ defined by $$x \cdot (A^{*} y) = (Ax) \cdot y$$ for all $x \in \R^{n}, y \in \R^{m}$. 
\end{enumerate}
\end{definition}

\begin{proposition} \label{prop:properties_linear} \hfill
\begin{enumerate}[i)]
\item Let $A: \R^{n} \to \R^{m}$ and $B: \R^{k} \to \R^{n}$ be linear maps. Then we have $A^{**} = A$ and $(A \circ B)^{*} = B^{*} \circ A^{*}$.
\item Let $S : \R^{n} \to \R^{n}$ be a symmetric linear map. Then $S^{*} = S$.
\item If $O : \R^{n} \to \R^{m}$ is an orthogonal linear map, then $n \le m$ and $O^{*} \circ O = I$ on $\R^{n}$. 
%\begin{align*}
%O^{*} \circ O = I & \text{ on } \R^{n}.
%%O \circ O^{*} = I & \text{ on } \R^{m}.
%\end{align*}
\end{enumerate}
\end{proposition}

\begin{theorem}[Polar decomposition] \label{thm:polar_decomposition_map}
Let $L: \R^{n} \to \R^{m}$ be a linear mapping.
\begin{enumerate}[i)]
\item If $n \le m$, there exists a symmetric map $S: \R^{n} \to \R^{m}$ and an orthogonal map $O: \R^{n} \to \R^{m}$ such that $$L = O \circ S.$$
\item If $n \ge m$, there exists a symmetric map $S: \R^{m} \to \R^{m}$ and an orthogonal map $O: \R^{m} \to \R^{n}$ such that $$L = S \circ O^{*}.$$
\end{enumerate}
\end{theorem}

\begin{definition}[Jacobian]
Assume $L : \R^{n} \to \R^{m}$ is linear.
\begin{enumerate}[i)]
\item If $n \le m$, we write $L = O \circ S$ as above, and we define the {\em Jacobian} of $L$ as $$\J L := |\det{S}|.$$ 
\item If $n \le m$, we write $L = S \circ O^{*}$ as above, and we define the {\em Jacobian} of $L$ as $$\J L := |\det{S}|.$$ 
\end{enumerate}
\end{definition}

In the literature, these two different definitions of Jacobian are also called {\em $n$-dimensional Jacobian} (or {\em area factor}), and {m}-dimensional {\em coarea factor}, respectively, and are denoted by $\J_{n}$ and ${\bf C}_{m}$.

\begin{theorem}[Representation of Jacobian] \hfill
\begin{enumerate}[i)]
\item If $n \le m$, $$\J L = \sqrt{\det{(L^{*} \circ L)}}.$$
\item If $n \ge m$, $$\J L = \sqrt{\det{(L \circ L^{*})}}.$$
\end{enumerate}
\end{theorem}
\begin{proof}
Let $n \le m$ and $L = O \circ S$, by Theorem \ref{thm:polar_decomposition_map}. Then we have $L^{*} = S \circ O^{*}$, so that $$L^{*} \circ L = S \circ O^{*} \circ O \circ S = S^{2},$$ since $O$ is orthogonal and so $O^{*} \circ O$ is the identity mapping on $\R^{n}$ (by Proposition \ref{prop:properties_linear}). Hence $$\det{(L^{*} \circ L)} = \det{S^{2}} = (\J L)^{2}.$$
The proof of (ii) is similar.
\end{proof}

\begin{remark}
The definition of the Jacobian of $L$ is independent of the choices of $O$ and $S$, and we have $\J L = \J L^{*}$.
\end{remark}

\begin{proposition}[Cauchy-Binet formula] \label{Cauchy_Binet}
If $n \le m$ and $L: \R^{n} \to \R^{m}$ is a linear map, then
\begin{equation*}
\J L = \sqrt{ \sum_{B} (\det{(B)})^{2}}
\end{equation*}
where the sum is taken over all $n\times n$ minor of any matrix representation of $L$.
\end{proposition}

Let now $f : \R^{n} \to \R^{m}$, $f = (f^{1}, \dots, f^{m}),$ be a Lipschitz map. By Rademacher's theorem (Theorem \ref{thm:Rademacher}), $f$ is differentiable $\Leb{n}$-a.e. and therefore the gradient matrix 
\[\nabla f(x) = \begin{pmatrix} \frac{\partial f^{1}}{\partial x_{1}} & \cdots & \frac{\partial f^{1}}{\partial x_{n}} & \\ \vdots & \ddots & \vdots \\  \frac{\partial f^{m}}{\partial x_{1}} & \cdots & \frac{\partial f^{m}}{\partial x_{n}} & \end{pmatrix}(x)\]
is well defined for $\Leb{n}$-a.e. $x \in \R^{n}$ and can be considered a linear map from $\R^{n}$ into $\R^{m}$.

\begin{definition}
If $f : \R^{n} \to \R^{m}$ is Lipschitz continuous and $x$ is a differentiability point, we define the {\em Jacobian} of $f$ as $$\J f(x) := \J \nabla f(x).$$
\end{definition}

\begin{remark}
Notice that $\J f \le c_n \Lip(f)^{n}$.
\end{remark}

\subsection{The area formula}

Through this subsection we assume $n \le m$ and $f : \R^{n} \to \R^{m}$ to be Lipschitz continuous.

\begin{lemma} Let $A \subset \R^{n}$ be $\Leb{n}$-measurable. Then
\begin{enumerate}[i)]
\item $f(A)$ is $\Haus{n}$-measurable,
\item the mapping $y \to \Haus{0}(A \cap f^{-1}(y))$ is $\Haus{n}$-measurable on $\R^{m}$ and
\begin{equation*}
\int_{\R^{m}} \Haus{0}(A \cap f^{-1}(y)) \, d \Haus{n}(y) \le (\Lip(f))^{n} \Leb{n}(A)
\end{equation*}
\end{enumerate}
\end{lemma}

\begin{definition}
The mapping $y \to \Haus{0}(A \cap f^{-1}(y))$ is the {\em multiplicity function} of $f$ in $A$.
\end{definition}

\begin{remark}
It is easy to notice that $\Haus{0}(A \cap f^{-1}(y))$ is equal to the cardinality of the set of $$\{ x \in A : f(x) = y \},$$
so that $f^{-1}(y)$ is finite for $\Haus{n}$-a.e. $y \in \R^{m}$.
In particular, if $f$ is injective, then
\begin{equation*}
\Haus{0}(A \cap f^{-1}(y)) = \begin{cases} 1 & y \in f(A), \\
0 & y \notin f(A).
\end{cases}
\end{equation*}
\end{remark}

\begin{theorem}[Area formula] \label{area_formula}
Let $f : \R^{n} \to \R^{m}$ be Lipschitz continuous and $n \le m$. Then, for all $\Leb{n}$-measurable sets $A \subset \R^{n}$, we have
\begin{equation} \label{eq:area_formula}
\int_{A} \J f(x) \, dx = \int_{\R^{m}} \Haus{0}(A \cap f^{-1}(y)) \, d\Haus{n}(y).
\end{equation}
\end{theorem}

This means that the $\Haus{n}$-measure of $f(A)$, counting multiplicity, is equal to the integral of the Jacobian of $f$ over $A$. As an immediate consequence, we deduce a generalization of the classical change of variables formula.

\begin{theorem}[General change of variables] \label{change_variables_gen}
Let $f : \R^{n} \to \R^{m}$ be Lipschitz continuous and $n \le m$. Then, for all $\Leb{n}$-summable functions $g : \R^{n} \to \R$, we have
\begin{equation} \label{eq:change_variables_gen}
\int_{\R^{n}} g(x) \, \J f(x) \, dx = \int_{\R^{m}} \left (\sum_{x \in f^{-1}(y)} g(x) \right ) d \Haus{n}(y).
\end{equation}
\end{theorem}

\begin{corollary}[Injective maps] \label{change_variables}
Let $f : \R^{n} \to \R^{m}$ be Lipschitz continuous and $n \le m$. Let $g : \R^{n} \to \R$ be a $\Leb{n}$-summable function, and assume that $f$ is injective on the support of $g$. Then, we have
\begin{equation} \label{eq:change_variables_1}
\int_{\R^{n}} g(x) \, \J f(x) \, dx = \int_{f(\R^{n})} g(f^{-1}(y)) \, d \Haus{n}(y).
\end{equation}
Equivalently, if $h : \R^{m} \to \R$ is such that $h \circ f$ is $\Leb{n}$-summable and $f$ is injective on the support of $h$, then we have
\begin{equation} \label{eq:change_variables_2}
\int_{\R^{n}} h(f(x)) \, \J f(x) \, dx = \int_{f(\R^{n})} h(y) \, d \Haus{n}(y).
\end{equation}
If $g = \chi_{A}$ for some $\Leb{n}$-measurable set $A$, then
\begin{equation} \label{eq:area_formula_inj}
\Haus{n}(f(A)) = \int_{A} \J f(x) \, dx.
\end{equation}
\end{corollary}

\begin{remark} \label{rem:other_good_cases}
Theorem \ref{change_variables_gen} and Corollary \ref{change_variables} hold also in the case $g : \R^{n} \to [0, + \infty]$ is $\Leb{n}$-measurable; however, the left hand sides of \eqref{eq:change_variables_gen} and \eqref{eq:change_variables_1} may be equal to $+ \infty$.
In addition, since any Borel function is Lebesgue measurable, Theorem \ref{change_variables_gen} and Corollary \ref{change_variables} are valid for all Borel functions $g : \R^{n} \to \R$ either nonnegative or $\Leb{n}$-summable.
\end{remark}

We list here some remarkable applications of the area formula.

\begin{example}[Length of a curve]
Let $n = 1, m \ge 1$. Assume $f : \R \to \R^{m}$ is Lipschitz and injective. It is clear that, for $\Leb{1}$-a.e. $t \in \R$,
\begin{equation*}
\J_{1} f(t) = |\dot{f}(t)|. 
\end{equation*}
Therefore, for any $a, b \in \R, a < b$, the length of a curve $C := f([a, b])$ is given by
\begin{equation*}
\Haus{1}(C) = \int_{a}^{b} |\dot{f}| \, dt,
\end{equation*}
thanks to \eqref{eq:area_formula_inj}.
\end{example}

\begin{example}[Surface area of a graph] \label{example:surface_area_graph}
Let $n \ge 1$ and $m = n + 1$. Assume $g: \R^{n} \to \R$ is Lipschitz and define $f: \R^{n} \to \R^{n + 1}$ as $$f(x) := (x, g(x)).$$
Then
\[\nabla f = \begin{pmatrix} 1 & \cdots & 0 & \\ \vdots & \ddots & \vdots \\ 
0 & \cdots & 1 & \\ \frac{\partial g}{\partial x_{1}} & \cdots & \frac{\partial g}{\partial x_{n}} & \end{pmatrix},\]
and so, by Cauchy-Binet formula (Proposition \ref{Cauchy_Binet}), we have
\begin{equation*}
\J f = \sqrt{1 + |\nabla g|^2}.
\end{equation*}
For any set $G \subset \R^{n}$, we define the graph of $g$ over $G$ as
$$ \Gamma(g, G) := \{ (x, g(x)) : x \in G \} \subset \R^{n + 1},$$
and we write $\Gamma(g) := \Gamma(g, \R^{n})$.
Therefore, if $G$ is $\Leb{n}$-measurable, \eqref{eq:area_formula_inj} yields
\begin{equation*}
\Haus{n}(\Gamma(g, G)) = \int_{G} \sqrt{1 + |\nabla g|^{2}} \, dx,
\end{equation*}
so that $\dim_{\Haus{}}\left ( \Gamma(g, G)\right ) = n$ for any $\Leb{n}$-measurable set $G$. Thus, we see that $\Haus{n} \res \Gamma(g)$ is a locally finite Radon measure on $\R^{n + 1}$. In addition, by \eqref{eq:change_variables_2} and Remark \ref{rem:other_good_cases}, for any Borel function $\varphi$ with compact support and any $\Leb{n}$-measurable set $G$ we have
\begin{equation} \label{eq:area_formula_surface}
\int_{\Gamma(g, G)} \varphi(y) \, d \Haus{n}(y) = \int_{G} \varphi(x, g(x)) \sqrt{1 + |\nabla g(x)|^{2}} \, d x.
\end{equation} 
\end{example}

\begin{example}[Surface area of a parametric hypersurface]
Let $n \ge 1$ and $m = n + 1$. Assume $f : \R^{n} \to \R^{n + 1}, f = (f^1, \dots, f^{n + 1}),$ is Lipschitz and injective. For any $k \in \{1, \dots, n + 1 \}$, we define $$\hat{f}_{k} := (f^{1}, \dots, f^{k - 1}, f^{k + 1}, \dots, f^{n + 1});$$
that is, the vector valued map $f$ without its $k$-th component. Then, it is not difficult to see that $$\J \hat{f}_{k} = |\det{\nabla \hat{f}_{k}}|,$$
and so, as a consequence of Cauchy-Binet formula (Proposition \ref{Cauchy_Binet}), we have
\begin{equation*}
\J f = \sqrt{\sum_{k = 1}^{n + 1} (\J \hat{f}_{k})^{2}}.
\end{equation*}
Thus, if we define $\Sigma(f, \Omega) := f(\Omega)$, for any open set $\Omega \subset \R^{n}$ to be a portiong of the parametric hypersurface, \eqref{eq:area_formula_inj} yields
\begin{equation*}
\Haus{n}(\Sigma(f, \Omega)) = \int_{\Omega} \sqrt{\sum_{k = 1}^{n + 1} (\J \hat{f}_{k})^{2}} \, dx.
\end{equation*}
\end{example}



\subsection{The Gauss--Green and integration by parts formulas on regular open sets}

We conclude this section by exploiting the area formula to prove the classical Gauss--Green formula on open sets with $C^1$-regular boundary.

\begin{definition} \label{def:C_k_boundary}
Let $E$ be an open set in $\R^{n}$ and $k \in \N \cup \{\infty\}$. We say that $E$ has {\em $C^{k}$ boundary} (or {\em smooth boundary} if $k = \infty$) if for all $x \in \partial E$ there exists $r > 0$ and $\psi \in C^{k}(B(x, r))$ with $\nabla \psi(y) \neq 0$ for all $y \in B(x, r)$ and 
\begin{align*}
B(x, r) \cap E & = \{ y \in B(x, r) : \psi(y) > 0 \}, \\
B(x, r) \cap \partial E & = \{ y \in B(x, r) : \psi(y) = 0 \}.
\end{align*}
We define the {\em inner unit normal to $E$}, $\nu_{E}(y)$ for $y \in \partial E$ as 
\begin{equation*}
\nu_{E}(y) := \frac{\nabla \psi(y)}{|\nabla \psi(y)|} \ \text{ for all } y \in B(x, r) \cap \partial E.
\end{equation*}
\end{definition}

It can be checked that this definition is independent from the choice of $\psi$ and $r$. Therefore, $\nu_{E}$ can be considered as a vector field defined on the whole of $\partial E$, and $\nu_{E} \in C^{k - 1}(\partial E; \mathbb{S}^{n - 1})$.

\begin{remark}\label{rem:local_param_C_1}
If $E$ is an open set with $C^{1}$ boundary, then $\Haus{n - 1} \res \partial E$ is a (locally finite) Radon measure on $\R^{n}$.
Indeed, we can see this by applying the implicit function theorem. We set 
\begin{equation*}
p: \R^{n} \to \R^{n - 1} \text{ and } q :\R^{n} \to \R
\end{equation*}
to be orthogonal projections, so that $x = (p(x), q(x))$, and, for any $s, r > 0$, let
\begin{equation*}
D(p(x), s) := \{ w \in \R^{n - 1} : |p(x) - w| < s \} 
\end{equation*}
and 
\begin{equation*}
C(x, r) := \{ y \in \R^{n} : |p(x - y)| < r, |q(x - y)| < r \}
\end{equation*}
to be a disk in $\R^{n - 1}$ and a cylinder in $\R^{n}$ centered in $x$, respectively. If $x \in \partial E$ and $r > 0$ satisfy the conditions of Definition \ref{def:C_k_boundary}, then there exists $s > 0$ and $u \in C^{1}(D(p(x), s))$ such that $C(x, s) \subset B(x, r)$ and, up to a rotation,
\begin{align} 
\begin{split} \label{eq:param_C_1}
C(x, s) \cap E & = \{ y \in C(x, s) : q(y) > u(p(y)) \}, \\
C(x, s) \cap \partial E & = \{ y \in C(x, s) : q(y) = u(p(y)) \}.
\end{split}
\end{align}
Then, following the notation of Example \ref{example:surface_area_graph}, we have
\begin{equation*}
\Haus{n - 1} \res C(x, s) \cap \partial E = \Haus{n - 1} \res \Gamma(u, D(p(x), s)),
\end{equation*}
so that this is a finite Radon measure, by the area formula \eqref{eq:area_formula_surface}. Thus, $\Haus{n - 1}\res \partial E$ is a locally finite Radon measure in $\R^{n}$, since we can cover $\partial E$ with countably many cylinders with the properties listed above.

In addition, it is possible to see that the inner unit normal can be written as
\begin{equation} \label{eq:inner_unit_param}
\nu_{E}(y)= \frac{(- \nabla u(p(y)), 1)}{\sqrt{1 + |\nabla u(p(y))|^2}}
\end{equation}
for any $y \in C(x, s) \cap \partial E$.
\end{remark}

\begin{theorem}[Gauss--Green formula] \label{thm:Gauss_Green_formula}
If $E$ is an open set with $C^{1}$ boundary, then for all $\varphi \in C^{1}_{c}(\R^{n})$ we have
\begin{equation} \label{eq:Gauss_Green_formula}
\int_{E} \nabla \varphi \,dx = - \int_{\partial E} \varphi \, \nu_{E} \, d \Haus{n - 1}.
\end{equation}
\end{theorem}

As an immediate consequence of this theorem, we have the following two corollaries: the divergence theorem and the integration by parts formula.

\begin{corollary}[Divergence theorem] \label{cor:divergence_theorem}
If $E$ is an open set with $C^{1}$ boundary, then for all $F \in C^{1}_{c}(\R^{n}; \R^{n})$ we have
\begin{equation} \label{eq:divergence_theorem}
\int_{E} \div F \,dx = - \int_{\partial E} F \cdot \nu_{E} \, d \Haus{n - 1}.
\end{equation}
\end{corollary}

\begin{corollary}[Integration by parts formula] \label{cor:integration_parts_formula}
If $E$ is an open set with $C^{1}$ boundary, then for all $F \in C^{1}_{c}(\R^{n}; \R^{n})$ and $\varphi \in C^{1}_{c}(\R^{n})$ we have
\begin{equation} \label{eq:integration_parts_formula}
\int_{E} \varphi \, \div F \, dx + \int_{E} F \cdot \nabla \varphi \, dx = - \int_{\partial E} \varphi \, F \cdot \nu_{E} \, d \Haus{n - 1}.
\end{equation}
\end{corollary}

In order to prove Theorem \ref{thm:Gauss_Green_formula} we need a preliminary result on Lipschitz functions.

\begin{proposition} \label{prop:weak_der_Lip_b}
If $f \in \Lip(\R^{n})$, then
\begin{equation} \label{eq:weak_der_Lip_b}
\int_{\R^{n}} f \, \div \varphi \, dx = - \int_{\R^{n}} \varphi \cdot \nabla f \, dx
\end{equation}
for all $\varphi \in C^{1}_{c}(\R^{n}; \R^{n})$.
\end{proposition}
\begin{proof}
Let $\nu \in \mathbb{S}^{n - 1}$ and define
\begin{equation*}
g_{t, \nu}(x) := \frac{f(x + t \nu) -f(x)}{t}
\end{equation*}
for $t > 0$. It is easy to see that $\|g_{t, \nu}\|_{L^{\infty}(\R^{n})} \le \Lip(f; \R^{n})$.
By Rademacher's theorem (Theorem \ref{thm:Rademacher}), we have that $\nabla f(x)$ exists for $\Leb{n}$-a.e. $x \in \R^{n}$, and it satisfies
\begin{equation*}
g_{t, \nu}(x) \to \nabla f(x) \cdot \nu
\end{equation*}
as $t \to 0^{+}$ for $\Leb{n}$-a.e. $x \in \R^{n}$. Then, for any $\psi \in C^{1}_{c}(\R^{n})$ we have
\begin{equation*}
\int_{\R^{n}} g_{t, \nu}(x) \, \psi(x) \, dx = \int_{\R^{n}} \frac{f(x + t\nu) -f(x)}{t} \, \psi(x) \, dx = \int_{\R^{n}} f(x) \, \frac{\psi(x - t \nu) - \psi(x)}{t} \, dx.
\end{equation*}
Therefore, by taking the limit as $t \to 0^{+}$, Lebesgue's dominated convergence theorem implies that
\begin{equation*}
\int_{\R^{n}} \nabla f(x) \cdot \nu \, \psi(x) \, dx = \int_{\R^{n}} f(x) \, \nabla \psi(x) \cdot (- \nu) \, dx.
\end{equation*}
Thus, by taking $\nu = {\rm e}_{j}$ for each $j \in \{1, \dots, n\}$, we obtain
\begin{equation*}
\int_{\R^{n}} \frac{\partial f}{\partial x_{j}} \, \psi \,dx = - \int_{\R^{n}} f \, \frac{\partial \psi}{\partial x_{j}} \, dx.
\end{equation*}
Finally, if $\varphi \in C^{1}_{c}(\R^{n}; \R^{n})$, we conclude that
\begin{equation*}
\int_{\R^{n}} \nabla f \cdot \varphi \, dx = \int_{\R^{n}} \sum_{j = 1}^{n} \frac{\partial f}{\partial x_{j}} \, \varphi_{j} \,dx = - \sum_{j = 1}^{n} \int_{\R^{n}} f \, \frac{\partial \varphi_{j}}{\partial x_{j}} \, dx = - \int_{\R^{n}} f  \, \div \varphi \, dx.
\end{equation*}
\end{proof}

\begin{proof}[Proof of Theorem \ref{thm:Gauss_Green_formula}]
Recall the definitions of the projections $p, q$, the disk $D(p(x), s)$ and the cylinder $C(x, r)$ given in Remark \ref{rem:local_param_C_1}.
Given $x \in \partial E$, up to a rotation, we can choose $r, s > 0$ and $u \in C^{1}(D(p(x), s))$ such that \eqref{eq:param_C_1} holds. We claim that
\begin{equation} \label{eq:claim_GG}
\int_{E} \nabla \varphi \, dx = - \int_{\partial E} \varphi \nu_{E} \, d \Haus{n - 1}
\end{equation}
for any $\varphi \in C^{1}_{c}(C(x, s))$. To show this, let $\delta > 0$ and define a Lipschitz function $f_{\delta} : \R^{n} \to \R$ by setting
\begin{equation*}
f_{\delta}(y) = \begin{cases} 1 & \text{ if } q(y) > u(p(y)) + \delta, \\
0 & \text{ if } q(y) < u(p(y)) - \delta, \\
\frac{q(y) - u(p(y)) + \delta}{2 \delta} & \text{ if } |q(y) - u(p(y))| \le \delta, \end{cases}
\end{equation*}
for any $y \in C(x, s)$, and then by extending it to $\R^{n}$ using McShane's lemma (Lemma \ref{McShane_lemma}). It is easy to see that $f_{\delta} \to \chi_{C(x, s) \cap E}$ in $L^{1}(C(x, s))$. Hence, we obtain
\begin{equation*}
\int_{E} \nabla \varphi \, dx = \lim_{\delta \to 0^{+}} \int_{C(x, s)} f_{\delta} \nabla \varphi \, d x = - \lim_{\delta \to 0^{+}} \int_{C(x, s)} \varphi \nabla f_{\delta} \, dx,
\end{equation*}
where in the last equality we used Proposition \ref{prop:weak_der_Lip_b}. Let
\begin{equation*}
F_{\delta} := \{ y \in C(x, s) : |q(y) - u(p(y))| \le \delta \}.
\end{equation*}
Then, we have $\nabla f_{\delta}(y) = 0$ for any $y \in C(x, s) \setminus F_{\delta}$ and 
\begin{equation*}
\nabla f_{\delta}(y) = \frac{1}{2 \delta} (- \nabla u(p(y)), 1) \ \text{ for any } y \in F_{\delta}.
\end{equation*}
By Fubini's theorem, we obtain
\begin{equation*}
\int_{C(x, s)} \varphi \nabla f_{\delta} \, dx = \int_{F_{\delta}} \varphi \nabla f_{\delta} \, dx = \int_{D(p(x), s)} (- \nabla u(p(y)), 1) \frac{1}{2 \delta} \int_{u(z) - \delta}^{u(z) + \delta} \varphi(z, t) \, dt \, dz.
\end{equation*}
Therefore, by exploiting the continuity of $\varphi$, we can pass to the limit as $\delta \to 0^+$, and, exploiting \eqref{eq:inner_unit_param} and the area formula \eqref{eq:area_formula_surface}, we obtain
\begin{align*}
\int_{E} \nabla \varphi \, dx & = - \lim_{\delta \to 0^{+}} \int_{C(x, s)} \varphi \nabla f_{\delta} \, dx \\
& = - \lim_{\delta \to 0^{+}} \int_{D(p(x), s)} (- \nabla u(p(y)), 1) \frac{1}{2 \delta} \int_{u(z) - \delta}^{u(z) + \delta} \varphi(z, t) \, dt \, dz \\
& = - \int_{D(p(x), s)} (- \nabla u(p(y)), 1) \varphi(z, u(z)) \, d z \\
& = - \int_{D(p(x), s)} \varphi(z, u(z)) \nu_{E}(z, u(z)) \sqrt{1 + |\nabla u(z)|^2} \, dz \\
& = - \int_{C(x, s) \cap \partial E} \varphi \nu_{E} \, d \Haus{n - 1} = - \int_{\partial E} \varphi \nu_{E} \, d \Haus{n - 1}.
\end{align*}
We conclude now the proof by a partition of unity argument. Let $\varphi \in C^{1}_c(\R^{n})$ and $A$ be an open bounded set such that $\supp \varphi \cap \partial E \subset A$. By compactness, there exist $\{x_{k}\}_{k = 1}^{N} \subset A \cap \partial E$ and $\{s_{k}\}_{k = 1}^{N} \subset (0, + \infty)$ such that 
\begin{itemize}
\item $C(x_k, s_k) \subset A$,
\item $\supp \varphi \cap \partial E \subset \bigcup_{k = 1}^{N} C(x_k, s_k)$,
\item $\displaystyle \int_{E} \nabla (\zeta_{k} \varphi) \, dx = - \int_{\partial E} \zeta_{k} \varphi \nu_{E} \, d \Haus{n - 1}$ for any $\zeta_{k} \in C^{1}_{c}(C(x_{k}, s_{k}))$.
\end{itemize}
Consider now a partition of unity $\{\zeta_k\}_{k = 1}^{N}$ satisfying 
\begin{equation*}
\zeta_{k} \in C^{1}_{c}(C(x_{k}, s_{k})), 0 \le \zeta_k \le 1 \text{ and } \sum_{k = 1}^{N} \zeta_{k}(x) = 1 \text{ for any } x \in A.
\end{equation*}
Then, we can find a cutoff function $\zeta_{0} \in C^{1}_{c}(E)$ such that $0 \le \zeta_{0} \le 1$ and $$\sum_{k = 0}^{N} \zeta_{k}(x) = 1 \text{ for any } x \in E \cap A.$$
It is clear that $\zeta_{0} \varphi \in C^{1}_{c}(E)$, so that
\begin{equation*}
\int_{E} \nabla(\zeta_{0} \varphi ) \, dx = 0.
\end{equation*}
Thus, we obtain
\begin{equation*}
\int_{E} \nabla \varphi \, dx = \sum_{k = 0}^{N} \int_{E} \nabla (\zeta_{k} \varphi) \, dx = - \sum_{k = 1}^{N} \int_{\partial E} \zeta_{k} \varphi \nu_{E} \, d \Haus{n - 1} = - \int_{\partial E} \varphi \nu_{E} \, d \Haus{n - 1}.
\end{equation*}
\end{proof}

We notice that it is possible to extend Theorem \ref{thm:Gauss_Green_formula} to a slightly larger family of integration domains.

\begin{definition}
We say that an open set $E$ in $\R^{n}$ has {\em almost $C^{1}$ boundary} if there exists a closed set $M_{0} \subset \partial E$ with $\Haus{n - 1}(M_{0}) = 0$ and such that $M:= \partial E \setminus M_{0}$ is $C^{1}$-regular. $M$ is called the {\em regular part of $\partial E$}, while $M_{0}$ is the {\em singular part}.
\end{definition}

\begin{theorem} \label{thm:Gauss_Green_almost_C_1}
Let $E$ be an open set with almost $C^{1}$ boundary. Then for all $\varphi \in C^{1}_{c}(\R^{n})$ we have
\begin{equation*}
\int_{E} \nabla \varphi \,dx = - \int_{M} \varphi \, \nu_{E} \, d \Haus{n - 1}.
\end{equation*}
\end{theorem}

\begin{remark}
Since the singular part of the boundary of a set with almost $C^1$ boundary is $\Haus{n - 1}$-negligible, we see that we actually have
\begin{equation*}
\int_{E} \nabla \varphi \,dx = - \int_{\partial E} \varphi \, \nu_{E} \, d \Haus{n - 1}
\end{equation*}
for any $\varphi \in C^{1}_{c}(\R^{n})$, up to setting $\nu_{E} \equiv {\rm e}_{1}$ on $M_{0}$, for instance.
\end{remark}







