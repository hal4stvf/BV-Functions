\setcounter{section}{-1}
\section{Motivation}
Let's  consider
\[
\inf\left\{\int_\Omega |\nabla u|dx \,:\, u \in W^{1,1}(\Omega), \|u\|_{L^1} = K
> 0 \right\} =: m_K,
\]
where $W^{1,1}(\Omega) := \{u \in L^1(\Omega) : Du \in L^1(\Omega, \R^n)\}$ is
the Sobolev space. Then there exists a sequence $(u_j \in
W^{1,1}(\Omega))_{j\in\N}$ such that $\|u_j\|_{L^1(\Omega)}=K$ and $\|\nabla
u_j\|_{L^1(\Omega)} \to m_K$ for $j \to \infty$. Now, in general this does
\emph{not} imply that there is a subsequence $(u_{j_k})_{k\in\N}$ which will
convergence even only weakly to an $v \in W^{1,1}(\Omega)$ with $\|v\|_{L^1} =
K$ and $\|\nabla v \|_{L^1} = m_k$. 

The reason for this is essentially because $L^1$ is not a (topological) dual of
any space, though it is contain in one. \todo{Lacks details}

Another example is the \emph{\textbf{Isoperimetric problem}}:
\[
\min\left\{\sigma_{n-1}(\partial F) \,:\, \text{$F$ with some regularity}, |F| =
K > 0 \right\} =: \gamma_k,
\]
where $|F| =: \mathcal{L}^n(F)$ denotes the $n$-dimensional Lebesgue measure.

\section{Measures}

Let $X$ be a non-empty set. We denote by $\mathcal{P}(X)$ (or $2^X$) the
\emph{power set}, that is, the collection of all subsets of $X$.

\begin{definition}[(outer) measure]
A mapping $\mu : \mathcal{P}(X) \to [0,+\infty]$ satisfying
\begin{enumerate}[(1)]
\item $\mu(\emptyset) = 0$ 
\item $\mu(A) \leq \sum_{k=1}^\infty \mu(A_k)$ if $A \subset
\bigcup_{k=1}^\infty A_k$ \hfill ($\sigma$-subadditivity)
\end{enumerate}
is called an (outer) measure.
\end{definition}

\begin{remark}
The (outer) measure is not decreasing, that is, for $A\subset B$, where $A,B \in
\mathcal{P}(X)$, we have $\mu(A) \leq \mu(B)$. 
\end{remark}

\begin{definition}[Restriction of a measure] If $Y \subset X$, the
\emph{restriction of $\mu$ to $Y$}, denoted by $\mu \mres Y$,
is defined as $(\mu \mres Y)(A) := \mu(Y\cap A)$.
\end{definition}

\begin{definition}[$\mu$-measurable] We call a subset $A \subset X$
$\mu$-measurable if 
\[
\mu(B) = \mu(B\cap A) + \mu(B \setminus A)
\qquad \text{for all} \quad B \subseteq X.
\]
\end{definition}

\begin{remark}
This definition is meaningful since \emph{Vitali} found that there exists a set
$E \subset \R$ which is \emph{not} $\mathcal{L}^1$-measurable.
\end{remark}

\begin{definition}[$\sigma$-algebra]
A subset $\mathfrak{F} \subset \mathcal{P}(X)$ is called a \emph{$\sigma$-algebra of
sets} if holds
\begin{enumerate}[(1)]
\item $\emptyset,X \in \mathfrak{F}$,
\item for $A \in \mathfrak{F}$ also $X\setminus A \in \mathfrak{F}$,
\item for a family $(A_i \in \mathfrak{F})_{i\in I}$ we have 
have $\bigcup_{i\in I} A_i \in \mathfrak{F}$.
\end{enumerate}
\end{definition}

\begin{theorem}
Let $\mu$ be a (outer) measure on $X$, then the restriction to the
$\sigma$-algebra of $\mu$-measurable sets is $\sigma$-additive, that is, if
$(A_j)_{j\in I}$ is a (at most) countable disjoint $\mu$-measurable family of
subsets of $X$, then 
\[
\mu\left(\bigcup_{j \in I} A_j\right) = \sum_{j\in I} \mu\left(A_j\right).
\]
\end{theorem}

\begin{definition} Here we collect some important definitions 
\begin{enumerate}[(1)]
%%%
\item Let $\mathfrak{C} \subset \mathcal{P}(X)$, we call the smallest
$\sigma$-algebra containing $\mathfrak{C}$, the \emph{$\sigma$-algebra generated by
$\mathfrak{C}$}.\footnote{Here $\mathfrak{C}=\setminus\text{mathfrak}\{C\}$} 
%%%
\item The \emph{Borel-algebra} on $\R^n$, denoted by $\mathcal{B}(\R^n)$, is the
$\sigma$-algebra generated by the family of open sets in $\R^n$ (in the standard
topology). The elements of the Borel-algebra are called \emph{Borel sets}.
%%%
\item A (outer) measure $\mu$ in $\R^n$ is called a \emph{Borel measure} if each
Borel sets is $\mu$-measurable.
\item A (outer) measure $\mu$ in $\R^n$ is called \emph{Borel regular} if for
all subsets $A \subseteq \R^n$ there exists a Borel set $B$ such that $A \subseteq
B$ and $\mu(A) = \mu(B)$.
\item A Borel regular measure $\mu$ which is locally finite (e.g. $\mu(K) <
\infty$ for all compact subsets $K \subset \R^n$), is called a \emph{Radon measure}.
\end{enumerate}
\end{definition}

\begin{theorem}
Let $\mu$ be a Radon measure on $\R^n$. We have
\begin{enumerate}[(1)]
\item  for all $A \subseteq \R^n$ holds $\mu(A) = \inf\left\{ \mu(U) \,:\,
U \supset A,\, U \text{ open}\right\}$ \hfill (outer regularity),
\item for all $\mu$-measurable sets $B$ holds 
$\mu(B) = \sup \left\{ \mu (K): K \subset B,\, K \text{ compact}\right\}$ \hfill
(inner regularity).
\end{enumerate}
\end{theorem}

\begin{theorem}[Carath\'eodory's criteria]
Let $\mu$ be a (outer) measure on $\R^n$. If for all $A,B \subset \R^n$ that
satisfy $\dist(A,B) > 0$ we have $\mu(A \cup B) = \mu(A) + \mu(B)$, then $\mu$
is a Borel measure.
\end{theorem}

\begin{examples}~
\begin{enumerate}[(1)]
\item For $x \in \R^n$ we can define that \emph{dirac measure} by  
\[
\delta_x(A) := 
\begin{cases}
1 & x \in A
\\
0 & x \not\in A.
\end{cases}
\]
This is in fact a Radon measure.
%%%
\item We define the \emph{counting measure} by
\[
\# (E) = 
\begin{cases}
\text{card}(E) & \text{if $E$ is finite}
\\
+\infty & \text{otherwise}.
\end{cases}
\]
This measure is Borel regular, but \emph{not} a Radon measure (since it is
clearly not locally finite).
\item The also have the well-known \emph{Lebesgue measure} defined by
\[
\mathcal{L}^n(A) := \inf \left\{\sum_{i=1}^\infty \mathcal{L}^n(Q_i) \mid A
\subset \bigcup_{i=1}^\infty Q_i,\, Q_i \text{ cubes}\right\},
\]
where $\mathcal{L}^n(Q_i)$ is equal to the side length of the cubes $Q_i$ to the
$n$-th power. In particular, we have
\[
\mathcal{L}^1(A) = \inf \left\{ \sum_{i=1}^\infty  \diam C_j \mid A
\subset \bigcup_{i=1}^\infty C_j, \, C_j \subset \R \right\}
\]
and so we can characterize
\[
\mathcal{L}^n = \underbrace{\mathcal{L}^1\times\mathcal{L}^1 \times \dots \times
\mathcal{L}^1}_{n-\text{times}} = \mathcal{L}^{n-1} \times \mathcal{L}^1.
\]
%%%%%
\item (\textbf{Hausdorff measure}) Consider $A \subseteq \R^n$, $\alpha \geq 0$, $\delta \in (0,+\infty]$, we
define the \emph{Hausdorff $\alpha$-dimensional content of $A$} as
\[
\mathcal{H}^\alpha_\delta(A) := \inf\left\{ \sum_{j\in I} \omega_\alpha
\left(\frac{\diam C_j}{2}\right)^\alpha \mid A \subset \bigcup_{j\in I \subset \N} C_j, \, \diam
C_j \leq \delta, C_j \subseteq \R^n \right\},
\]
where the infimum is taking over all the (at most countable) coverings $(C_j \subset
\R^n)_{j\in I}$ of $A$, and set $$\omega_\alpha :=
\frac{\pi^{\frac{\alpha}{2}}}{\Gamma\left(\frac{\alpha}{2}+1\right)}.$$ Since
$\mathcal{H}^\alpha_\delta$ is a not-increasing function in $\delta$ the
following limit
\[
\mathcal{H}^\alpha(A) := \lim_{\delta \searrow 0} \mathcal{H}^\alpha_\delta (A) =
\sup_{\delta > 0} \mathcal{H}^\alpha_\delta(A)
\]
always exists in the extended real numbers. This limit is defined to be the
\emph{Hausdorff measure}.
\end{enumerate}
\end{examples}

\begin{theorem}[Hausdorff measure is Borel regular]
$\mathcal{H}^\alpha$ is a Borel regular measure on $\R^n$ for all $\alpha \geq
0$.
\end{theorem}

\begin{theorem}[Basic properties of the Hausdorff measure]~
\begin{enumerate}[(1)]
\item $\mathcal{H}^0 = \#$ 
\item $\mathcal{H}^1 = \mathcal{L}^1$ on $\R$ 
\item $\mathcal{H}^\alpha \equiv 0$ for all $\alpha > n$ in $\R^n$. 
\item $\mathcal{H}^\alpha(\lambda A) = \lambda^\alpha \mathcal{H}^\alpha(A)$ for
all $A \subseteq \R^n$ and $\lambda > 0$
\item $\mathcal{H}^\alpha(L(A)) = \mathcal{H}^\alpha(A)$ for all affine
isometry $L : \R^n \to \R^n$.
\end{enumerate}
\end{theorem}

\begin{proof}~
\begin{enumerate}[(1)]
\item Since $\omega_0 = 1$ we have 
\[
\begin{aligned}
\mathcal{H}^0(A) &= \lim_{\delta \searrow 0} \inf \left\{ \sum_{j\in I}
\left(\frac{\diam(C_j)}{2}\right)^0 \mid A \subset \bigcup_{j\in I \subset \N}
C_j, \,
\diam C_j \leq \delta  \right\} 
\\& =
\lim_{\delta \searrow 0} \inf \left\{ \sum_{j\in I}
1 \mid A \subset \bigcup_{j\in I} C_j,\,
\diam C_j \leq \delta  \right\} 
\\&=
\begin{cases}
\text{card}(A) & \text{if $A$ is finite}
\\
\infty & \text{otherwise}
\end{cases}
\end{aligned}
\]
%%
\item We estimate the Lebesgue measure $\mathcal{L}^1$ from both sides by the
Hausdorff measure: Since $\omega_1 = 2 = |(-1,1)|$ we first get 
\[
\begin{aligned}
\mathcal{L}^1(A) 
&= \inf \left\{ \sum_{j\in I} \diam C_j \mid A \subset \bigcup_{j\in I} C_j \right\}
\\ & \leq 
\inf \left\{ \sum_{j\in I} \diam C_j \mid A \subset \bigcup_{j\in I} C_j, \,
\diam C_j \leq \delta \right\}
= 
\mathcal{H}^1_\delta(A),
\end{aligned}
\]
which is true for all $\delta > 0$ so we obtained $\mathcal{L}^1(A) \leq
\mathcal{H}^1(A)$.
\\
Now, we define a partition of $\R$ by $J_{k,\delta} := [k\delta, (k+1)\delta]$
for $k \in \Z$ and first fixed $\delta >0$. These are intervals of diameter
$\delta$ so for every $j \in I$ we get $\diam(C_j \cap I_{k,\delta}) \leq
\delta$. Also we have $\sum_{k\in\Z} \diam(C_j \cap I_{k,\delta}) \leq \diam C_j$,
since $I_{k,\delta}$ are a partition $\R$ of disjoint intervals in $k$. So we
get
\[
\begin{aligned}
\mathcal{L}^1(A) = \inf \left\{ \sum_{j\in I} \diam C_j \mid A \subset
\bigcup_{j\in I} C_j \right\}
\geq \inf \left\{ \sum_{j\in I} \sum_{k\in\Z} \diam (C_j \cap I_{k,\delta}) \mid A \subset
\bigcup_{j\in I} \bigcup_{k\in\Z} C_j\cap I_{k,\delta} \right\}
\end{aligned}
\]
since $\diam (C_j \cap I_{k,\delta} ) \leq \delta$ and after relabeling the
index sets $I$ and $\Z$ to an index set $I^{(k,\delta)}$ this last expressions reads 
\[
\dots = \inf \left\{ \sum_{j \in I^{(k,\delta)}} C_j^{(k)} \mid A \subset \bigcup_{j \in
I^{(k,\delta)}} C_j^{(k)},\, \diam C_j^{(k)} \leq \delta \right\} \geq \mathcal{H}^1_\delta.
\]
And since this is true for every $\delta >0$ we arrive at the claim.
\item Let $\alpha > n$ and $Q$ be a unit cube in $\R^n$. If we consider cubes
$Q_i$ with side length $\frac{1}{m}$ for any fixed $m\in \N$, we get $\diam Q_i
= \frac{\sqrt{n}}{m}$ and $Q \subset \bigcup_{i=1}^{m^n} Q_i$. From this we can 
infer
\[
\begin{aligned}
\mathcal{H}_{\frac{\sqrt{n}}{m}}^\alpha (Q) 
& \leq \sum_{j=1}^{m^n} \omega_\alpha \left(\frac{\diam Q_i}{2}\right)^\alpha
= \frac{\omega_\alpha}{2^\alpha} \sum_{j=1}^{m^n}
\left(\frac{\sqrt{n}}{m}\right)^\alpha = \frac{\omega_\alpha}{2^\alpha}
n^{\frac{\alpha}{2}} m^{n-\alpha}
%\qquad \searrow 0 \quad \text{as} \quad m \to \infty.
\end{aligned}
\]
and read off with the crucial assumptions $ n < \alpha$
\[
\begin{aligned}
\mathcal{H}^\alpha (Q) = \lim_{m\to \infty} \mathcal{H}_{\frac{\sqrt{n}}{m}}^\alpha (Q) 
= \frac{\omega_\alpha}{2^\alpha}
\lim_{m\to \infty} n^{\frac{\alpha}{2}} m^{n-\alpha}
= 0.
\end{aligned}
\]
This is the claim, since $\R^n$ can be covered by a countable collection of unit cubes.
\end{enumerate}
\end{proof}

\begin{lemma}
Let $A \subset \R^n$ and $\delta > 0$ such that $\mathcal{H}^\alpha_\delta = 0$,
then follows that $\mathcal{H}^\alpha(A) = 0$.
\end{lemma}

\begin{proof}
Since the Hausdorff content is non-increasing in $\delta$, we have $0 =
\mathcal{H}^\alpha_\delta(A) \geq \mathcal{H}^\alpha_\infty(A)$. That
convergence means that for every $\varepsilon > 0$ there exists a (at most
countable) family of subsets $(C_j)_{j\in I}$ such that 
\[
A \subseteq
\bigcup_{j\in I} C_j 
\quad \text{and} \quad
\sum_{j\in I} \omega_\alpha \left(\frac{\diam C_j}{2} \right)^\alpha <
\varepsilon.
\]
For this to be true the diameter of every $C_j$ must be controlled by $\diam C_j
\leq 2 \left(\frac{\varepsilon}{\omega_\alpha}\right)^{\frac{1}{\alpha}} = :
\delta_\varepsilon$. So we have $\mathcal{H}^\alpha_{\delta_\varepsilon} \leq
\varepsilon$ and as $\delta_\varepsilon \searrow 0$ so will $\varepsilon \searrow
0$. But this is the claim $\mathcal{H}^\alpha (A) = 0$.
\end{proof}

\begin{proposition}
Let $A \subseteq \R^n$, $0 \leq s < t < \infty$.
\begin{enumerate}[(1)]
\item If $\mathcal{H}^s(A) < \infty$, then $\mathcal{H}^t(A) = 0$. 
\item If $\mathcal{H}^t(A) > 0$, then $\mathcal{H}^s(A) = +\infty$.
\end{enumerate}
\end{proposition}
\begin{proof}
\begin{enumerate}[(1)]
\item Fix $\delta > 0$ and a (at most countable) family of subsets 
$(C_j)_{j\in I}$ such that 
\[
\diam C_j \leq \delta 
\quad \text{and} \quad
\sum_{j\in I} \omega_s
\left(\frac{\diam C_j}{2}\right)^s \leq \mathcal{H}^s_\delta(A) + 1 \leq
\mathcal{H}^s(A) + 1.
\]
From this follows
\[
\begin{aligned}
\mathcal{H}^t_\delta(A) \leq \sum_{j\in I} \omega_t \left(\diam C_j\right)^t 
&= \frac{\omega_t}{\omega_s} s^{s-t} \sum_{j\in I} \omega_s 
\left(\frac{\diam C_j}{2} \right)^s (\diam C_j)^{t-s}
\\ &\leq C_{s,t} \delta^{t-s} \left(\mathcal{H}^s(A) +1 \right) 
\, \longrightarrow 0 \quad \text{as} \quad \delta \to 0,
\end{aligned}
\]
which is the claim $\mathcal{H}^t(A) = 0$.
\item If by contradiction $\mathcal{H}^s(A) < \infty$, then by $(1)$ if follows
that $\mathcal{H}^r(A) = 0$ for all $ r > s$ and such in particular $r = t$.
\end{enumerate}
\end{proof}

\begin{definition}
We call the \emph{Hausdorff dimension} of a set $A\subset \R^n$
\[
\dim_{\mathcal{H}}(A) := \inf\left\{ \alpha \geq 0 : \mathcal{H}^\alpha(A) = 0
\right\}.
\]
\end{definition}

\begin{remark}
\todo{@GC please check and edit}
Let $\alpha = \dim_{\mathcal{H}}(A)$. Then one has
\[
\mathcal{H}^s(A) = 0 \quad \text{for all} \quad s > \alpha
\qquad \text{and also} \qquad
\mathcal{H}^t(A) = +\infty \quad \text{for all} \quad t < \alpha,
\]
where the first equality is clear from the definition of the Hausdorff dimension
and the second fact follows like this: Suppose by contradiction
$\mathcal{H}^t(A) < \infty$ for some $t < \alpha$, then by the
above proposition we have $\mathcal{H}^r(A) = 0$ for all $ r > t$. But this
means it even holds for all $r \in (t,\alpha)$ and this is a  
contradiction to the fact that $\alpha$ is indeed the infimum. \todo{@GC Is this
the right argument?}
\\ 
In particular $\mathcal{H}^\alpha$ is \emph{not} a Radon measure for all $\alpha \in
[0,n)$. Take for example the closed unit Ball $\overline{B(0,1)}$ in $\R^n$. We
know that $0 < \mathcal{H}^n(\overline{B(0,1)}) < \infty$ and so
$\mathcal{H}^\alpha(\overline{B(0,1)}) = +\infty$ for all $\alpha < n$.
\end{remark}

If a Borel set $E \subseteq \R^n$ satisfies $\mathcal{H}^\alpha(E) \in
(0,\infty)$, then $\mathcal{H}^\alpha \mres E$ is a Radon measure. Indeed in
general we have the following

\begin{theorem}
If $\mu$ is a Borel regular measure in $\R^n$ and $A \subset \R^n$ is
$\mu$-measurable and $\mu(A) < \infty$, then $\mu \mres A$ is a Radon measure. 
\end{theorem}

\begin{theorem}
$\mathcal{H}^n = \mathcal{L}^n$
\end{theorem}

\begin{proposition}
Let $\alpha \geq 0$, $A \subset \R^n$.
\begin{enumerate}[(1)]
\item If $f : \R^n \to \R^m$ is Lipschitz, then $\mathcal{H}^\alpha(f(A)) \leq
(\Lip(f))^\alpha \mathcal{H}^\alpha(A)$.
\item If $g : \R^n \to \R^m$ is $\gamma$-Hölder\footnote{E.g. $|g(x) - g(y)| \leq
C_\gamma |x-y|^\gamma$ for a $\gamma \in (0,1)$.}, then $\mathcal{H}^\alpha(g(A)) \leq
C_{\alpha,\gamma} \mathcal{H}^\alpha(A)$.
\end{enumerate}
\end{proposition}

\begin{proof}[Proof of (2)]
Fix $\delta > 0$, and take a (at most countable) family of subsets $(C_j)_{j\in I}$
such that $ A \subset \bigcup_{j\in I} C_j$ and $\diam C_j \leq \delta$.
For $g(C_j)$ we get for all $j\in I$
\[
\begin{aligned}
\diam g(C_j)\leq C \left(\diam C_j\right)^\gamma \leq C \delta^\gamma
\quad \text{and} \quad
g(A) \subseteq \bigcup_{j\in I} g(C_j).
\end{aligned}
\]
Using this, we can infer
\[
\begin{aligned}
\mathcal{H}^\alpha_{C \delta^\gamma}(g(A)) \leq \sum_{j\in I} \omega_\alpha
\left(\frac{\diam g(C_j)}{2}\right)^\alpha
\leq 
\underbrace{\frac{\omega_\alpha}{2^\alpha} \frac{C^\alpha
2^{\alpha\gamma}}{\omega_{\alpha\gamma}}}_{=: C_{\alpha,\gamma}} \sum_{j\in I} \omega_{\alpha \gamma} 
\left(\frac{\diam C_j}{2}\right)^{\alpha\gamma}
\end{aligned}
\]
%\vspace{-10em}
and by taking the infimum over all coverings $(C_j)_{j\in I}$ we get
\[
\mathcal{H}^\alpha_{C\delta^\gamma} (g(A)) \leq C_{\alpha,\gamma}
\mathcal{H}^{\alpha\gamma}_\delta (A)
\]
and by sending $\delta \searrow 0 $ we arrive at the claim.
\end{proof}

Silyinksi triangle (Wachar Siejinski 1915).

Construction a fractal of triangles in triangles. Specifically take $S_k$ is the
union of $s^k$ equilateral triangles with side length $2^{-k}$.
\[
S = \bigcup_{k=0}^\infty S_k
\qquad
\mathcal{H}^\alpha_{\frac{1}{2^k}} \leq \sum_{j=1}^{3^k}
\frac{\omega_\alpha}{2^\alpha} ( \diam S_k^j)^\alpha 
= \frac{\omega_\alpha}{2^\alpha} 3^k 2^{-k\alpha} \searrow 0 \quad as \quad k\to
\infty \quad iff \quad \alpha > \frac{\log^3}{\log^2}
\]
\[
\Rightarrow \mathcal{H}^\alpha(S) = 0 \quad \forall \alpha > \frac{\log 3}{\log
2} \Rightarrow \dim_{\mathcal{H}}(S) \leq \frac{\log 3}{\log 2}
\]

\begin{theorem}
$\mathcal{H}^n = \mathcal{L}^n$ on $\R^n$.
\end{theorem}

\begin{lemma}[Vitali covering property for $\mathcal{L}$]
For all open $U$ and for all $\delta > 0$ there exists a family of disjunct
closed balls $(\overline{B_K})_{k=1}^\infty$ such that $\diam B_k < \delta$ and
$\mathcal{L}^n(U \setminus \bigcup_{k=1}^\infty \overline{B_k}) = 0$ 
\end{lemma}

\begin{theorem}[isodiametric inequality]
For all $E \subset \R^n$ Lebesgue measurable we have 
\[
|E| \leq \omega_n \left(\frac{\diam E}{2}\right)^n.
\]
\end{theorem}

\begin{proof}[Proof of first theorem]
\TODO
\begin{enumerate}[{Step} 2]
\item $\mathcal{L}^n(A) \leq \mathcal{H}^n(A)$ for all $A \subset \R^n$ 
Fix $\delta > 0$. Let $(C_j)_{j\in I}$: $A \subset \bigcup_{j\in I} C_j$, $\diam
C_j \leq \delta$. From this follows 
\[
\mathcal{L}^n(A) \leq \sum_{j=1}^\infty \mathcal{L}^n(C_j) \leq
\sum_{j=1}^\infty \omega_n \left(\frac{\diam C_j}{2}\right)^n,
\]
where in the last inequality we used the \emph{isometric inequality}. Taking the
infimum over all $(C_j)$ we get 
\[
\mathcal{L}^n(A) \leq \mathcal{H}^n_\delta (A) \qquad \text{for all} \quad
\delta > 0.
\]
\item $\mathcal{H}^n_{\delta} \leq C_n \mathcal{L}^n$ for some $C_n \geq 1$
\[
\mathcal{L}^n(A) 
= \inf\left\{ \sum_{j=1}^\infty \mathcal{L}^n(Q_j) \mid A
\subset \bigcup Q_j\right\}
\geq \inf\left\{ \sum_{j=1}^\infty \mathcal{L}^n(Q_j) \mid A
\subset \bigcup Q_j, \diam Q_j < \delta \right\}
= \frac{2^n}{(\sqrt{n})^n \omega_n} \inf
\left\{\sum_{j=1}^\infty \omega_n (\frac{\diam Q_j}{2})^2 \mid 
A \subset \bigcup Q_j, \diam Q_j < \delta\right\}
\geq \frac{1}{C_n} \mathcal{H}^n_\delta (A)
\]
\item By definition of $\mathcal{L}^n$, for all $\delta, \varepsilon > 0$, there
exists $(Q_i)_{j=1}^\infty$ such that $A \subset \bigcup_{j=1}^\infty Q_j$,
$\diam Q_j \leq \delta$ and $\sum_{j=1}^\infty \mathcal{L}^n(Q_j) \leq
\mathcal{L}^n(A) + \varepsilon$. There exists $(B_j^i)_{i=1}^\infty$ disjoint
closed balls such that $B_j^i \subset Q_j$ for all $(\diam B^i_j \subset
\delta)$ and $\mathcal{L}^n(\circ Q^j \setminus \bigcup_{i=1}^\infty
\overline{B}_j^i) = 0 = \mathcal{L}^n(Q_j \setminus \bigcup_{i=1}^\infty$
\end{enumerate}
\end{proof}

\begin{proof}[Proof of isodiametric inequality]
If $E \subset B(x, \frac{\diam E}{2})$ for some $x$ then it's trivial.
WLOG, $E$ is compact. $\diam A = \diam \overline{A}$.
Steiner symmetrization (1838). Decompose $\R^n = \R^{n-1}\times \R^1$ and let $p
: \R^n \to \R^{n-1}$, $q: \R^n \to \R$ so that $x = (px, qx)$, $q(x) = x_n$
\[
\forall x \in \R^{n-1} \quad E_z := \left\{t\in\R : (z,t) \in E\right\}
\quad \textnormal{vertical section}
\]
define
\[
E^s := \left\{ x \in \R^n: |q(x)| \leq \frac{\mathcal{L}^1(E(p(x)))}{2}\right\}
\]
By Fubini's theorem, $E_z$ is $\mathcal{L}^1$-measurable for
$\mathcal{L}^{n-1}$-a.e. $z$, $z \mapsto \mathcal{L}^1(E_z)$ is Lebesgue
measurable 
\[
|E| = \int_{\R^{n-1}} \mathcal{L}^1(E_z) dz = \int_{\R^{n-1}}
\mathcal{L}^1(E_z^s) dz = |E^s|
\]
where the first equal sign is a \TODO(nebenrechnung), and the second equality
follows with Fubini.
\parbox{1\linewidth}{
$\mathcal{L}^1(E^s_z) = \mathcal{L}^1(E_z)$	
\[
E^s_z = 
\]
}
\\ %
Now we claim $\diam E^s \leq \diam E$. Let $x \in E^s$ and define $M(x), m(x)
\in E$ to be the points for which 
\[
\begin{aligned}
p(m(x)) &= p(M (x)) = px
\\
q(m(x)) &= q(z) \leq q(M(x))
\qquad \text{for all} \quad z \in E \quad \text{with} \quad p(z) = p(x).
\end{aligned}
\]
Let $x,y \in E^s$, 
\[
|q(x) - q(y)| \leq \max \left\{|q(M(x)) - q(m(y))|, |q(M(y)) - q(m(x))|\right\}
\overset{WLOG}{=} |q(M(x)) - q(m(y))|
\]
\[
|x-y|^2 = |p(x-y)|^2 + |q(x-y)|^2 \leq \max \left\{|M(x) - m(y)|, |M(y) -
m(x)|\right\}^2
\leq (\diam E)^2 .
\]
From this follows $|x - y| \leq \diam E$ for all $x,y \in E^s$.

Given a $\mathcal{L}^n$ measurable set $F$, we define $F^i$ to be the Steiner
symmetrization with respect to the $i$-th coordinate axis.
$E_0 := E$, $E_i := (E_{i=1})^i$ with $i \in \{1,2,\dots,n\}$. Then $|E_n| = |E|$,
$\diam E_n \leq \diam E$ and, if $x\in E_n$, then $-x \in E_n$. From this
follows $E_n \subset B\left(0,\frac{\diam E_n}{2}\right)$.
And with this we are done!
\end{proof}

\section{Integration}

Let $X \neq \emptyset$, $\mu$ be a measure on $X$.

\begin{definition}
\begin{enumerate}[(1]
\item A function $u : X \to [-\infty,\infty]$ $\overline{R}$ is $\mu$-measurable if
$\{u > t\} = \{ x \in X : u(x) > t\}$ is $\mu$-measurable for all $t\in \R$.
\item $u$ is a $\mu$-simple function is it is $\mu$-measurable and $u(X)$ is
countable (that is $u(x) = \sum_{k=1} u_k \chi_{E_k}(x)$)
\item If $u$ is a non-negative $\mu$-simple function, we define 
\[
\int_X u d\mu := \sum_{t \in u(X)} t\mu(\{u=t\}) = \sum_{k=1}^\infty u_k
\mu(E_k) \in [0,\infty]
\]
where $0 \cdot \infty = 0$.
\item Set $u^\pm := \max\{\pm u, 0 \}$, $u = u^+ - u^-$. 
If $u$ is $\mu$-simple and $\int_X u^+ d\mu$ or $\int_X u^- d\mu < \infty$, then
\[
\int_X u d\mu := \int_X u^+ d\mu - \int_X \mu^- d\mu \in [-\infty,\infty]
\]
If $v$ satisfies (4), is called $\mu$-integrable simple function.
\item If $u$ is $\mu$-measurable, we define the upper and lower integrals of $u$
as
\item \TODO
\end{enumerate}
\TODO
\end{definition}
