\section{General measures}

Let $X$ be a non-empty set. We denote by $\mathcal{P}(X)$ (or $2^X$) the
\emph{power set}; that is, the collection of all subsets of $X$.

\begin{definition}[Measures] \label{def:measure}
A mapping $\mu : \mathcal{P}(X) \to [0,+\infty]$ satisfying
\begin{enumerate}[(1)]
\item $\mu(\emptyset) = 0$, 
\item $\mu(A) \leq \sum_{k=1}^\infty \mu(A_k)$ if $A \subset
\bigcup_{k=1}^\infty A_k$ ($\sigma$-subadditivity),
\end{enumerate}
is called a measure.
\end{definition}

It should be noticed that in the literature a mapping as the one in Definition \ref{def:measure} is also called an {\em outer measure}, while the name of measure is used to denote the restriction of the mapping to the family of measurable set (see Definition \ref{def:measurable_set} below). We shall nevertheless follow the notation of \cite{evans2015measure}, in order to be able to assign a measure even to nonmeasurable sets.

\begin{remark}
Thanks to $\sigma$-subadditivity, any measure is not decreasing; that is, for $A\subset B$, where $A,B \in \mathcal{P}(X)$, we have $\mu(A) \leq \mu(B)$. 
\end{remark}

\begin{definition}[Restriction of a measure] If $Y \subset X$, the
\emph{restriction of $\mu$ to $Y$}, denoted by $\mu \mres Y$,
is defined as $(\mu \mres Y)(A) := \mu(Y\cap A)$ for any $A \subset X$.
\end{definition}

\begin{definition}[$\mu$-measurable sets] \label{def:measurable_set} 
We call a subset $A \subset X$
$\mu$-measurable if 
\[
\mu(B) = \mu(B\cap A) + \mu(B \setminus A)
\qquad \text{for all} \quad B \subseteq X.
\]
\end{definition}

\begin{remark}
This definition is meaningful, since the italian mathematician \emph{Giuseppe Vitali} proved in 1905 that there exists a set
$E \subset \R$ which is \emph{not} $\Leb{1}$-measurable \cite{vitali1905sul}. For a modern presentation of his construction, we refer to \cite[Section I.1.2]{maggi2012sets}.
\end{remark}

\begin{definition}[$\sigma$-algebra]
A subset $\mathfrak{F} \subset \mathcal{P}(X)$ is called a \emph{$\sigma$-algebra of
sets} if the following conditions hold:
\begin{enumerate}[(1)]
\item $\emptyset,X \in \mathfrak{F}$,
\item for any $A \in \mathfrak{F}$ we have $X\setminus A \in \mathfrak{F}$,
\item for any countable family of sets $\{A_i\}_{i \in I}$ such that $A_{i} \in \mathfrak{F}$ for any $i \in I$ we have 
have $$\bigcup_{i\in I} A_i \in \mathfrak{F}.$$
\end{enumerate}
\end{definition}

\begin{theorem}
Given any measure $\mu$ on $X$, the family of $\mu$-measurable sets forms a $\sigma$-algebra.
\end{theorem}

\begin{theorem}
Let $\mu$ be a measure on $X$, then the restriction to the
$\sigma$-algebra of $\mu$-measurable sets is $\sigma$-additive, that is, if
$(A_j)_{j\in I}$ is a countable disjoint $\mu$-measurable family of
subsets of $X$, then 
\[
\mu\left(\bigcup_{j \in I} A_j\right) = \sum_{j\in I} \mu\left(A_j\right).
\]
\end{theorem}

We list now some relevant definitions.

\begin{definition} \hfill
\begin{enumerate}[(1)]
%%%
\item Given any $\mathfrak{C} \subset \mathcal{P}(X)$, we call the smallest
$\sigma$-algebra containing $\mathfrak{C}$, the \emph{$\sigma$-algebra generated by
$\mathfrak{C}$}.%\footnote{Here $\mathfrak{C}=\setminus\text{mathfrak}\{C\}$} 
%%%
\item The \emph{Borel $\sigma$-algebra} on $\R^n$, denoted by $\mathcal{B}(\R^n)$, is the
$\sigma$-algebra generated by the family of open sets in $\R^n$ (in the standard
topology). The elements of the Borel $\sigma$-algebra are called \emph{Borel sets}.
%%%
\item A measure $\mu$ in $\R^n$ is called a \emph{Borel measure} if each
Borel sets is $\mu$-measurable.
\item A measure $\mu$ in $\R^n$ is called \emph{Borel regular} if for
all subsets $A \subseteq \R^n$ there exists a Borel set $B$ such that $A \subseteq
B$ and $\mu(A) = \mu(B)$.
\item A Borel regular measure $\mu$ which is locally finite (i.e. $\mu(K) <
\infty$ for all compact subsets $K \subset \R^n$), is called a \emph{Radon measure}.
\end{enumerate}
\end{definition}

\begin{theorem}
Let $\mu$ be a Radon measure on $\R^n$. Then we have
\begin{enumerate}[(1)]
\item $\mu(A) = \inf\left\{ \mu(U) \,:\,
U \supset A,\, U \text{ open}\right\}$ for all $A \subseteq \R^n$ \hfill (outer regularity),
\item $\mu(B) = \sup \left\{ \mu (K): K \subset B,\, K \text{ compact}\right\}$ for all $\mu$-measurable sets $B$ \hfill
(inner regularity).
\end{enumerate}
\end{theorem}

\begin{theorem}[Carath\'eodory's criterion] \label{caratheodory_criterion}
Let $\mu$ be a measure on $\R^n$. If for all $A,B \subset \R^n$ such that $\dist(A,B) > 0$ we have $$\mu(A \cup B) = \mu(A) + \mu(B),$$ then $\mu$
is a Borel measure.
\end{theorem}

Not any Borel regular measure is a Radon measure. However, it is possible to obtain a Radon measure as a restriction of a Borel regular one, as stated in the followin theorem.

\begin{theorem} \label{thm:Borel_restriction_Radon}
If $\mu$ is a Borel regular measure in $\R^n$ and $A \subset \R^n$ is
$\mu$-measurable and $\mu(A) < \infty$, then $\mu \mres A$ is a Radon measure. 
\end{theorem}

\begin{example}[Dirac delta]
For $x \in \R^n$ we define the \emph{Dirac\footnote{Named after Paul Adrien Maurice Dirac (1902-1984), English theoretical physicist who shared the 1933 Nobel Prize in Physics with Erwin Schrödinger "for the discovery of new productive forms of atomic theory". He actually introduced the so-called {\em Dirac delta function} as a "convenient notation" in his influential 1930 book {\em The Principles of Quantum Mechanics}. The name "delta function" was chosen since it works like a continuous analogue of the discrete Kronecker delta 
\[
\delta_{i \, j} := 
\begin{cases}
1 & \text{if} \, i = j,
\\
0 & \text{if} \, i \neq j.
\end{cases}
\]
Indeed, for any sequence $\{a_{j}\}_{j \in \Z}$, we have
\begin{equation*}
\sum_{j = -\infty}^{\infty} a_{j} \delta_{i \, j} = a_{i},
\end{equation*}
and, analogously, for any $x \in \R^{n}$ and any continuous function $f : \R \to \R$, the Dirac delta satisfies the property
\begin{equation*}
\int_{- \infty}^{+ \infty} f(y) \delta(x - y) \, dy = \int_{- \infty}^{\infty} f(y)\, d \delta_{x}(y) = f(x).
\end{equation*}
} measure centered in $x$} by setting
\[
\delta_x(A) := 
\begin{cases}
1 & x \in A,
\\
0 & x \not\in A.
\end{cases}
\]
It is easy to check that this is indeed a Radon measure. In addition, any set in $\R^{n}$ is $\delta_{x}$-measurable.
\end{example}

\begin{example}[The counting measure] 
We define the \emph{counting measure} by setting
\[
\# (E) = 
\begin{cases}
\text{card}(E) & \text{if $E$ is finite,}
\\
+\infty & \text{otherwise}.
\end{cases}
\]
This measure is Borel regular, but \emph{not} a Radon measure, since it is
clearly not locally finite.
\end{example}

\begin{example}[The Lebesgue measure] 
The well-known \emph{Lebesgue measure} is defined by
\[
\Leb{n}(A) := \inf \left\{\sum_{i=1}^\infty \Leb{n}(Q_i) \mid A
\subset \bigcup_{i=1}^\infty Q_i,\, Q_i \text{ cubes}\right\},
\]
where $\Leb{n}(Q_i) = ( l(Q_{i})^{n}$ and $l(Q_{i})$ is the side length of the cube $Q_i$. It is actually possible to show that in one dimension we have
\[
\Leb{1}(A) = \inf \left\{ \sum_{ij=1}^\infty  \diam C_j \mid A
\subset \bigcup_{i=1}^\infty C_j, \, C_j \subset \R \right\}
\]
and that we can characterize $\Leb{n}$ in an alternative way as
\[
\Leb{n} = \underbrace{\Leb{1}\times\Leb{1} \times \dots \times \Leb{1}}_{n-\text{times}} = \Leb{n-1} \times \Leb{1}.
\]
\end{example}


\section{The Hausdorff measure}

\begin{definition}[Hausdorff content] Consider $A \subseteq \R^n$, $\alpha \geq 0$, $\delta \in (0,+\infty]$, we
define the \emph{$\alpha$-dimensional Hausdorff content} of $A$ as
\[
\Haus{\alpha}_\delta(A) := \inf\left\{ \sum_{j\in I} \omega_\alpha
\left(\frac{\diam C_j}{2}\right)^\alpha \mid A \subset \bigcup_{j\in I \subset \N} C_j, \, \diam
C_j \leq \delta, C_j \subseteq \R^n \right\},
\]
where the infimum is taken over all the (at most countable) {\em $\delta$-coverings} $\{C_j\}_{j\in I}$ of $A$, and we set $$\omega_\alpha :=
\frac{\pi^{\frac{\alpha}{2}}}{\Gamma\left(\frac{\alpha}{2}+1\right)}.$$ 
\end{definition}

We notice that $\Haus{\alpha}_\delta(A)$ is a non-decreasing function in $\delta$, so that we can take the limit as $\delta \searrow 0$ and it always exists in the extended real numbers. This justifies the following definition.

\begin{definition}[Hausdorff measure]
For any $A \subset \R^{n}$ and $\alpha \ge 0$, we define the \emph{$\alpha$-dimensional Hausdorff measure} of $A$ as 
\[
\Haus{\alpha}(A) := \lim_{\delta \searrow 0} \Haus{\alpha}_\delta (A) =
\sup_{\delta > 0} \Haus{\alpha}_\delta(A).
\]
\end{definition}

Roughly speaking, we take the limit as $\delta \searrow 0$ since it forces the coverings to follow the local geometry of the set $A$. Indeed, the key idea behind the definition of the Hausdorff measure is that it should be able to capture the properties of thin sets in $\R^{n}$ (in particular, Lebesgue negligible sets). As we shall see in the following, if $\alpha = k \in \{ 1, \dots , n - 1\}$, then $\Haus{k}$ agrees with the $k$-dimensional surface area on sufficiently regular sets, as for instance $k$-dimensional planes.

It is not too difficult to prove that, as a consequence of Carath\'eodory's criterion, Theorem \ref{caratheodory_criterion}, any Borel set is $\Haus{\alpha}$-measurable, for any $\alpha \ge 0$.

\begin{theorem}[Hausdorff measures are Borel regular]
$\Haus{\alpha}$ is a Borel regular measure on $\R^n$ for all $\alpha \geq
0$.
\end{theorem}

\begin{theorem}[Basic properties of the Hausdorff measure]~
The following statements hold true:
\begin{enumerate}[(1)]
\item $\Haus{0} = \#$; 
\item $\Haus{1} = \Haus{1}_{\delta} = \Leb{1}$ on $\R$, for any $\delta > 0$; 
\item $\Haus{\alpha} \equiv 0$ for all $\alpha > n$ in $\R^n$;
\item $\Haus{\alpha}(\lambda A) = \lambda^\alpha \Haus{\alpha}(A)$ for all $A \subseteq \R^n$ and $\lambda > 0$;
\item $\Haus{\alpha}(L(A)) = \Haus{\alpha}(A)$ for all affine
isometry $L : \R^n \to \R^n$.
\end{enumerate}
\end{theorem}

\begin{proof}~
\begin{enumerate}[(1)]
\item Since $\omega_0 = 1$, we have $\Haus{0}_{\delta}(\{x\}) = 1$ for every $x \in \R^{n}$ and $\delta > 0$. Indeed, 
$$\omega_{0} \left(\frac{\diam(C_j)}{2}\right)^0 = 1,$$
which implies $\Haus{0}_{\delta}(\{x\}) \ge 1$, and, on the other hand, we can clearly cover the singleton only with itself. Hence, $\Haus{0}(\{x\}) = 1$ for every $x \in \R^{n}$. Since $\Haus{0}$ is a Borel measure, it is $\sigma$-additive on Borel sets, so that
$$ \Haus{0}(A) = \sum_{x \in A} \Haus{0}(\{x\}) = \# A, $$
for any finite or countable set $A$. Finally, if $A$ is incountable, then $A$ contains a countable set $B$, and so $\Haus{0}(A) \ge \Haus{0}(B) = + \infty$.
%\[
%\begin{aligned}
%\Haus{0}(A) &= \lim_{\delta \searrow 0} \inf \left\{ \sum_{j\in I}
%\left(\frac{\diam(C_j)}{2}\right)^0 \mid A \subset \bigcup_{j\in I \subset \N}
%C_j, \,
%\diam C_j \leq \delta  \right\} 
%\\& =
%\lim_{\delta \searrow 0} \inf \left\{ \sum_{j\in I}
%1 \mid A \subset \bigcup_{j\in I} C_j,\,
%\diam C_j \leq \delta  \right\} 
%\\&=
%\begin{cases}
%\text{card}(A) & \text{if $A$ is finite}
%\\
%\infty & \text{otherwise}
%\end{cases}
%\end{aligned}
%\]
%%
\item We estimate the Lebesgue measure $\Leb{1}$ from both sides by the
Hausdorff measure. Since $\omega_1 = 2 = |(-1,1)|$, for any $\delta > 0$ we first get 
\[
\begin{aligned}
\Leb{1}(A) 
&= \inf \left\{ \sum_{j\in I} \diam C_j \mid A \subset \bigcup_{j\in I} C_j \right\}
\\ & \leq 
\inf \left\{ \sum_{j\in I} \diam C_j \mid A \subset \bigcup_{j\in I} C_j, \,
\diam C_j \leq \delta \right\}
= 
\Haus{1}_\delta(A),
\end{aligned}
\]
%which is true for all $\delta > 0$ so we obtained $\Leb{1}(A) \leq \Haus{1}(A)$.
Now, we define a partition of $\R$ by setting $J_{k,\delta} := [k\delta, (k+1)\delta]$
for $k \in \Z$. These are intervals of diameter
$\delta$, so that, for every $j \in I$, we get 
\begin{equation} \label{eq:diam_J_k_delta}
\diam(C_j \cap J_{k,\delta}) \leq \delta.
\end{equation}
In addition, we have 
\begin{equation} \label{eq:diam_sum_control}
\sum_{k\in\Z} \diam(C_j \cap J_{k,\delta}) \leq \diam C_j,
\end{equation}
since $\{J_{k,\delta}\}_{k \in \Z}$ is a partition $\R$ of essentially disjoint intervals, because $\# (J_{k, \delta} \cap J_{m, \delta}) \le 1$ for any $k \neq m$. Therefore, by \eqref{eq:diam_sum_control} we get
\begin{align*}
\Leb{1}(A) & = \inf \left\{ \sum_{j\in I} \diam C_j \mid A \subset
\bigcup_{j\in I} C_j \right\} \\
& \geq \inf \left\{ \sum_{j\in I} \sum_{k\in\Z} \diam (C_j \cap J_{k,\delta}) \mid A \subset
\bigcup_{j\in I} \bigcup_{k\in\Z} C_j\cap J_{k,\delta} \right\}.
\end{align*}
We set now $C_{j} \cap J_{k, \delta} =: \tildef{C}_{i_{j, k}}$, by relabeling the indexes sets $I$ and $\Z$ to an index set $\tildef{I}$. Thanks to \eqref{eq:diam_J_k_delta}, we have $\diam(\tildef{C}_{i}) \le \delta$ and so we get
\begin{align*}
\Leb{1}(A) & \ge \inf \left\{ \sum_{j \in \tildef{I}} \tildef{C}_{i} \mid A \subset \bigcup_{i \in \tildef{I}} \tildef{C}_{i},\, \diam \tildef{C}_i \leq \delta \right\} \geq \Haus{1}_\delta(A).
\end{align*}
All in all, we get $\Leb{1} = \Haus{1}_{\delta}$ for any $\delta > 0$, from which it easily follows $\Leb{1} = \Haus{1}$ on $\R$.
\item Let $\alpha > n$ and $Q$ be a unit cube in $\R^n$. It is easy to see that, for any fixed $m\in \N$, $Q$ can be covered by $m^{n}$ smaller cubes $Q_i$ with side length $\frac{1}{m}$. Clearly, we have $\diam Q_i
= \frac{\sqrt{n}}{m}$. Therefore, we obtain 
\[
\begin{aligned}
\Haus{\alpha}_{\frac{\sqrt{n}}{m}} (Q) 
& \leq \sum_{j=1}^{m^n} \omega_\alpha \left(\frac{\diam Q_i}{2}\right)^\alpha
= \frac{\omega_\alpha}{2^\alpha} \sum_{j=1}^{m^n}
\left(\frac{\sqrt{n}}{m}\right)^\alpha = \frac{\omega_\alpha}{2^\alpha}
n^{\frac{\alpha}{2}} m^{n-\alpha},
%\qquad \searrow 0 \quad \text{as} \quad m \to \infty.
\end{aligned}
\]
from which we deduce that, since $n < \alpha$,
\[
\begin{aligned}
\Haus{\alpha} (Q) = \lim_{m\to \infty} \Haus{\alpha}_{\frac{\sqrt{n}}{m}} (Q) 
\le \frac{\omega_\alpha}{2^\alpha} n^{\frac{\alpha}{2}}
\lim_{m\to \infty} m^{n-\alpha}
= 0.
\end{aligned}
\]
Thus, the claim easily follows, since $\R^n$ can be covered by a countable collection of unit cubes and $\Haus{n}$ is $\sigma$-subadditive.
\end{enumerate}
The proofs of (4) and (5) are left as an exercise.
\end{proof}

\begin{lemma}
Let $A \subset \R^n$ and $\delta_{0} > 0$ such that $\Haus{\alpha}_{\delta_{0}}(A) = 0$,
then we have $\Haus{\alpha}(A) = 0$.
\end{lemma}

\begin{proof}
Since the Hausdorff content is non-increasing in $\delta$, we have $\Haus{\alpha}_\infty(A) \le \Haus{\alpha}_\delta(A)$ for any $\delta > 0$. In particular, this means that $\Haus{\alpha}_{\infty}(A) \le \Haus{\alpha}_{\delta_{0}}(A) = 0$, so that, for every $\varepsilon > 0$, there exists a countable family of sets $\{C_j\}_{j\in I}$ such that 
\[
A \subseteq
\bigcup_{j\in I} C_j 
\quad \text{and} \quad
\sum_{j\in I} \omega_\alpha \left(\frac{\diam C_j}{2} \right)^\alpha <
\varepsilon.
\]
In particular, the second condition immediately implies $$\diam C_j
\leq 2 \left(\frac{\varepsilon}{\omega_\alpha}\right)^{\frac{1}{\alpha}} = :
\delta_\varepsilon.$$ 
Hence, we have $\Haus{\alpha}_{\delta_\varepsilon} \leq
\varepsilon$, and $\delta_\varepsilon \searrow 0$ if and only if $\varepsilon \searrow
0$. This implies the claim $\Haus{\alpha}(A) = 0$.
\end{proof}

\begin{proposition} \label{prop:prop_Hausdorff_dim}
Let $A \subseteq \R^n$, $0 \leq s < t < \infty$.
\begin{enumerate}[(1)]
\item If $\Haus{s}(A) < \infty$, then $\Haus{t}(A) = 0$. 
\item If $\Haus{t}(A) > 0$, then $\Haus{s}(A) = +\infty$.
\end{enumerate}
\end{proposition}
\begin{proof}
\begin{enumerate}[(1)]
\item Fix $\delta > 0$ and a countable family of subsets 
$\{C_j\}_{j\in I}$ such that 
\[
\diam C_j \leq \delta 
\quad \text{and} \quad
\sum_{j\in I} \omega_s
\left(\frac{\diam C_j}{2}\right)^s \leq \Haus{s}_\delta(A) + 1 \leq
\Haus{s}(A) + 1.
\]
From this, it follows that
\[
\begin{aligned}
\Haus{t}_\delta(A) \leq \sum_{j\in I} \omega_t \left(\frac{\diam C_j}{2}\right)^t 
&= \frac{\omega_t}{\omega_s} 2^{s-t} \sum_{j\in I} \omega_s 
\left(\frac{\diam C_j}{2} \right)^s (\diam C_j)^{t-s}
\\ &\leq C_{s,t} \delta^{t-s} \left(\Haus{s}(A) +1 \right) 
\, \longrightarrow 0 \quad \text{as} \quad \delta \to 0,
\end{aligned}
\]
which implies the claim $\Haus{t}(A) = 0$.
\item If by contradiction $\Haus{s}(A) < \infty$, then by $(1)$ if follows
that $\Haus{r}(A) = 0$ for all $ r > s$ and in particular for $r = t$, which is clearly absurd.
\end{enumerate}
\end{proof}

\begin{definition}
We call the \emph{Hausdorff dimension} of a set $A\subset \R^n$ the number
\[
\dim_{\Haus{}}(A) := \inf\left\{ \alpha \geq 0 : \Haus{\alpha}(A) = 0
\right\}.
\]
\end{definition}

\begin{remark}
Let $\alpha = \dim_{\Haus{}}(A)$. Then one has
\begin{equation} \label{eq:H_dim_consequence}
\Haus{s}(A) = 0 \quad \text{for all} \quad s > \alpha
\qquad \text{and} \qquad
\Haus{t}(A) = +\infty \quad \text{for all} \quad t < \alpha.
\end{equation}
The first part of \eqref{eq:H_dim_consequence} follows clearly from the definition of the Hausdorff dimension. The second, instead, can be proved by contradiction. Suppose by contradiction that $\Haus{t}(A) < \infty$ for some $t < \alpha$, then, by the Proposition \ref{prop:prop_Hausdorff_dim}, we have $\Haus{r}(A) = 0$ for all $ r > t$. This implies 
$$\alpha = \inf\left\{ \beta \geq 0 : \Haus{\beta}(A) = 0
\right\} \le t < \alpha,$$
which is clearly absurd.

It should be noticed that, in general, $\Haus{\alpha}(A)$ can be any number in $[0, + \infty]$.
\end{remark}

We state now an important result on the equivalence between the Lebesgue measure on $\R^{n}$ and the $n$-dimensional Hausdorff measure, whose proof we postpone to the end of the section.

\begin{theorem} \label{thm:equivalence_Haus_Leb}
$\Haus{n}_{\delta} = \Haus{n} = \Leb{n}$ on $\R^{n}$, for any $\delta > 0$.
\end{theorem}

\begin{remark}
As a consequence of Theorem \ref{thm:equivalence_Haus_Leb}, we see that $\Haus{\alpha}$ is \emph{not} a Radon measure for all $\alpha \in
[0,n)$. Indeed, it is not bounded on some compact sets. Take for example the closed unit ball $\overline{B(0,1)}$ in $\R^n$. We know that $0 < \Haus{n}(\overline{B(0,1)}) < \infty$ and so, by Proposition \ref{prop:prop_Hausdorff_dim},
$\Haus{\alpha}(\overline{B(0,1)}) = +\infty$ for all $\alpha < n$.
\end{remark}

Even though $\Haus{\alpha}$ is not a Radon measure for $\alpha \in [0, n)$, it is possible to show that its restriction to some suitable Borel set is indeed a Radon measure.

\begin{proposition}
If a Borel set $E \subseteq \R^n$ satisfies $\Haus{\alpha}(E) \in
(0,\infty)$, then $\Haus{\alpha} \mres E$ is a Radon measure.
\end{proposition}
\begin{proof}
It is a simple consequence of Theorem \ref{thm:Borel_restriction_Radon}. 
\end{proof}

We investigate now the behaviour of the Hausdorff measure under the action of Lipschitz and H\"older functions. We recall first the definition of such family of functions.

\begin{definition}[Lipschitz and H\"older functions] Let $\Omega \subset \R^{n}$ be an open set.
\begin{enumerate}[(1)]
\item We say that $f : \Omega \to \R^{m}$ is {\em Lipschitz continuous} if there exists a constant $C > 0$ such that
\begin{equation} \label{eq:Lip_def}
|f(x) - f(y)| \le C |x - y| \ \ \text{for any} \ x, y \in \Omega.
\end{equation}
The smallest constant for which \eqref{eq:Lip_def} holds is called the {\em Lipschitz constant} of $f$, denoted by $\Lip(f)$ and alternatively characterized by
\begin{equation}
\Lip(f) := \sup \left \{ \frac{|f(x) - f(y)|}{|x - y|} \mid x, y \in \Omega, x \neq y \right \}.
\end{equation}
\item Let $\gamma \in (0,1)$. We say that $f: \Omega \to \R^{m}$ is {\em $\gamma$-H\"older continuous} if there exists a constant $C > 0$ such that
\begin{equation} \label{eq:Holder_def}
|f(x) - f(y)| \le C |x - y|^{\gamma} \ \ \text{for any} \ x, y \in \Omega.
\end{equation}
\end{enumerate}
\end{definition}

From this point on, we shall refer to Lipschitz continuous and H\"older continuous functions simply as Lipschitz and H\"older functions. 

\begin{exercise}
Show that any Lipschitz or $\gamma$-H\"older function (for some $\gamma \in (0,1)$) is indeed continuous.
\end{exercise}

\begin{remark} \label{rem:Lip_Holder}
Lipschitz functions can be seen as $1$-H\"older functions. Indeed, for any open set $\Omega \subset \R^{n}$ and any $\gamma \in [0, 1]$, we can define the space $C^{0, \gamma}(\Omega; \R^{m})$ of $\gamma$-H\"older functions as the set of continuous functions $f : \Omega \to \R^{m}$ for which there exists a constant $C > 0$ such that \eqref{eq:Holder_def} holds. If $\gamma = 0$, we have $C^{0, 0}(\Omega; \R^{m}) = C^{0}(\Omega; \R^{m})$. 
\end{remark}

\begin{exercise}
Let $\gamma > 1$ and $f: \Omega \to \R^{m}$ be such that there exists a constant $C > 0$ such that \eqref{eq:Holder_def} holds. Show that $f$ is constant.
\end{exercise}

\begin{proposition}
Let $\alpha \geq 0$, $A \subset \R^n$.
\begin{enumerate}[(1)]
\item If $f : \R^n \to \R^m$ is Lipschitz, then $\Haus{\alpha}(f(A)) \leq
(\Lip(f))^\alpha \Haus{\alpha}(A)$.
\item If $f : \R^n \to \R^m$ is $\gamma$-H\"older, for some $\gamma \in (0, 1)$, then $\Haus{\alpha}(f(A)) \leq
C_{\alpha,\gamma} \Haus{\alpha \gamma}(A)$.
\end{enumerate}
\end{proposition}

\begin{proof}
Thanks to Remark \ref{rem:Lip_Holder}, it is enough to prove (2) for any $\gamma \in (0, 1]$. Fix $\delta > 0$, and take a countable family of sets $\{C_j\}_{j\in I}$
such that $A \subset \bigcup_{j\in I} C_j$ and $\diam C_j \leq \delta$. It is clear that
$$f(A) \subseteq \bigcup_{j\in I} f(C_j).$$
Thanks to \eqref{eq:Holder_def}, we see that $f(C_j)$ satisfies
\begin{equation*}
\diam f(C_j)\leq C \left(\diam C_j\right)^\gamma \leq C \delta^\gamma,
\end{equation*}
where $C = \Lip(f)$ is $\gamma = 1$.
Hence, we obtain
\[
\begin{aligned}
\Haus{\alpha}_{C \delta^\gamma}(f(A)) \leq \sum_{j\in I} \omega_\alpha
\left(\frac{\diam f(C_j)}{2}\right)^\alpha
\leq 
\underbrace{\frac{\omega_\alpha}{2^\alpha} \frac{C^\alpha
2^{\alpha\gamma}}{\omega_{\alpha\gamma}}}_{=: C_{\alpha,\gamma}} \sum_{j\in I} \omega_{\alpha \gamma} 
\left(\frac{\diam C_j}{2}\right)^{\alpha\gamma}
\end{aligned}
\]
%\vspace{-10em}
and by taking the infimum over all $\delta$-coverings $\{C_j\}_{j\in I}$ we get
\[
\Haus{\alpha}_{C\delta^\gamma} (g(A)) \leq C_{\alpha,\gamma}
\Haus{\alpha\gamma}_\delta (A),
\]
where $C_{\alpha, \gamma} = \Lip(f)^{\alpha}$, if $\gamma = 1$.
By sending $\delta \searrow 0$ we conclude the proof.
\end{proof}
